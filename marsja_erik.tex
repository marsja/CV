\documentclass[]{article}
\usepackage{lmodern}
\usepackage{amssymb,amsmath}
\usepackage{ifxetex,ifluatex}
\usepackage{fixltx2e} % provides \textsubscript
\ifnum 0\ifxetex 1\fi\ifluatex 1\fi=0 % if pdftex
  \usepackage[T1]{fontenc}
  \usepackage[utf8]{inputenc}
\else % if luatex or xelatex
  \ifxetex
    \usepackage{mathspec}
    \usepackage{xltxtra,xunicode}
  \else
    \usepackage{fontspec}
  \fi
  \defaultfontfeatures{Mapping=tex-text,Scale=MatchLowercase}
  \newcommand{\euro}{€}
\fi
% use upquote if available, for straight quotes in verbatim environments
\IfFileExists{upquote.sty}{\usepackage{upquote}}{}
% use microtype if available
\IfFileExists{microtype.sty}{%
\usepackage{microtype}
\UseMicrotypeSet[protrusion]{basicmath} % disable protrusion for tt fonts
}{}
\usepackage[margin=1in]{geometry}
\ifxetex
  \usepackage[setpagesize=false, % page size defined by xetex
              unicode=false, % unicode breaks when used with xetex
              xetex]{hyperref}
\else
  \usepackage[unicode=true]{hyperref}
\fi
\hypersetup{breaklinks=true,
            bookmarks=true,
            pdfauthor={},
            pdftitle={Erik Marsja},
            colorlinks=true,
            citecolor=blue,
            urlcolor=blue,
            linkcolor=magenta,
            pdfborder={0 0 0}}
\urlstyle{same}  % don't use monospace font for urls
\usepackage{longtable,booktabs}
\usepackage{graphicx,grffile}
\makeatletter
\def\maxwidth{\ifdim\Gin@nat@width>\linewidth\linewidth\else\Gin@nat@width\fi}
\def\maxheight{\ifdim\Gin@nat@height>\textheight\textheight\else\Gin@nat@height\fi}
\makeatother
% Scale images if necessary, so that they will not overflow the page
% margins by default, and it is still possible to overwrite the defaults
% using explicit options in \includegraphics[width, height, ...]{}
\setkeys{Gin}{width=\maxwidth,height=\maxheight,keepaspectratio}
\setlength{\parindent}{0pt}
\setlength{\parskip}{6pt plus 2pt minus 1pt}
\setlength{\emergencystretch}{3em}  % prevent overfull lines
\providecommand{\tightlist}{%
  \setlength{\itemsep}{0pt}\setlength{\parskip}{0pt}}
\setcounter{secnumdepth}{0}

%%% Use protect on footnotes to avoid problems with footnotes in titles
\let\rmarkdownfootnote\footnote%
\def\footnote{\protect\rmarkdownfootnote}

%%% Change title format to be more compact
\usepackage{titling}

% Create subtitle command for use in maketitle
\newcommand{\subtitle}[1]{
  \posttitle{
    \begin{center}\large#1\end{center}
    }
}

\setlength{\droptitle}{-2em}
  \title{Erik Marsja}
  \pretitle{\vspace{\droptitle}\centering\huge}
  \posttitle{\par}
  \author{}
  \preauthor{}\postauthor{}
  \predate{\centering\large\emph}
  \postdate{\par}
  \date{2016-04-11}

\usepackage{titling}
\usepackage{caption}
\usepackage{tabularx}
\usepackage{booktabs}
\usepackage{tikz}
\usepackage{xcolor}
\usepackage{fancyhdr}
\usepackage{lastpage}
\usepackage{lscape}
\pagestyle{fancy}
\usepackage{titlesec}
\fancyhf{}
\renewcommand{\headrulewidth}{0pt}
\renewcommand{\footrulewidth}{0.4pt}
\fancyfoot[CO,CE]{CV - Erik Marsja}
\fancyfoot[LE,RO]{\thepage{/}\pageref*{LastPage}}
\fancypagestyle{plain}{\pagestyle{fancy}}
\newcommand{\blandscape}{\begin{landscape}}
\newcommand{\elandscape}{\end{landscape}}
\captionsetup[table]{labelformat=empty}


\titleformat{\section}
{\color{red}\normalfont\Large\bfseries}
{\color{red}\thesection}{1em}{}
\titleformat{\subsection}
{\color{red}\normalfont\Large\bfseries}
{\color{red}\thesubsection}{1em}{}

% Redefines (sub)paragraphs to behave more like sections
\ifx\paragraph\undefined\else
\let\oldparagraph\paragraph
\renewcommand{\paragraph}[1]{\oldparagraph{#1}\mbox{}}
\fi
\ifx\subparagraph\undefined\else
\let\oldsubparagraph\subparagraph
\renewcommand{\subparagraph}[1]{\oldsubparagraph{#1}\mbox{}}
\fi

\begin{document}
\maketitle

\begin{longtable}[c]{@{}lll@{}}
\toprule
\begin{minipage}[b]{0.41\columnwidth}\raggedright\strut
\textbf{Telephone}: +46 90 786 79 59
\strut\end{minipage} &
\begin{minipage}[b]{0.27\columnwidth}\raggedright\strut
\textbf{Mobile}: +46 70363 36 62
\strut\end{minipage} &
\begin{minipage}[b]{0.27\columnwidth}\raggedright\strut
\textbf{Email}:
\href{mailto:erik.marsja@umu.se}{\nolinkurl{erik.marsja@umu.se}}
\strut\end{minipage}\tabularnewline
\midrule
\endhead
\begin{minipage}[t]{0.41\columnwidth}\raggedright\strut
\textbf{University adress}:
\strut\end{minipage} &
\begin{minipage}[t]{0.27\columnwidth}\raggedright\strut
\strut\end{minipage} &
\begin{minipage}[t]{0.27\columnwidth}\raggedright\strut
\textbf{Home address}:
\strut\end{minipage}\tabularnewline
\begin{minipage}[t]{0.41\columnwidth}\raggedright\strut
Department of Psychology
\strut\end{minipage} &
\begin{minipage}[t]{0.27\columnwidth}\raggedright\strut
\strut\end{minipage} &
\begin{minipage}[t]{0.27\columnwidth}\raggedright\strut
Kandidatvägen
\strut\end{minipage}\tabularnewline
\begin{minipage}[t]{0.41\columnwidth}\raggedright\strut
Umeå University, SE-901 87, Umeå, Sweden
\strut\end{minipage} &
\begin{minipage}[t]{0.27\columnwidth}\raggedright\strut
\strut\end{minipage} &
\begin{minipage}[t]{0.27\columnwidth}\raggedright\strut
SE-907 33 Umeå, Sweden
\strut\end{minipage}\tabularnewline
\bottomrule
\end{longtable}

\subsection{Current research}\label{current-research}

\hrule

I am examining how sudden and unexpected changes in our environment
captures our attention and, thus, causes distraction from what we
currently are engaged in. Specifically, my Ph.D.~work is examining how
sudden and unexpected changes in auditory, tactile, and the combination
of both auditory and tactile (multisensory) stimulation affects
performance in speeded and short-term memory tasks.

\subsection{Research ambition}\label{research-ambition}

\hrule

My goal is to create my academic career in cognitive and applied
psychology. Specifically, in the areas of attention and cognitive
control. I aspire to eventually be able to work as an established
researcher in these, and related (i.e., cognitive neuroscience), areas.

\subsection{Education and Degrees}\label{education-and-degrees}

\hrule

\begin{itemize}
\item
  2012-Present Ph.D.~Student in Psychology
\item
  2011-2012 M.Sc. in Cognitive Science (\emph{Thesis}: Attention
  Capture: The Impact of Change in Spatial Sound Source on Behavior-
  Department of Psychology, Umeå University.
\item
  2008-2011 B.Sc. in Cognitive Science (\emph{Thesis}: Attention
  Capture: Studying the Distracting Effect of One's Own Name. Department
  of Psychology, Umeå University. Available from DiVA Database
  (\url{http://urn.kb.se/resolve?urn=urn:nbn:se:umu:diva-46607})
\end{itemize}

\subsection{Publications}\label{publications}

\subsubsection{International peer-reviewed
journals}\label{international-peer-reviewed-journals}

\hrule

Ljungberg, K. J., Parmentier, F. B. R., Jones, D. M., \textbf{Marsja},
E., \& Neely, G. (2014). ``What's in a name?'' ``No more than when it's
mine own2. Evidence from auditory oddball distraction. Acta
Psychologica, 150C, 161--166. \url{doi:10.1016/j.actpsy.2014.05.009}

\subsubsection{Publications In Preperation/Submitted/Under
Revision}\label{publications-in-preperationsubmittedunder-revision}

\hrule

\textbf{Marsja}, E., Neely, G., Ljungberg K.J. (Submitted). The impact
of presence and absence of standard vibrations and deviating sounds in a
visual attention task

Sörqvist, P., Marsh, J., \textbf{Marsja}, E., \& Ljungberg, K. J.
(Submitted). Irrelevant vibro-tactile stimuli produce a changing-state
effect: Implications for theories of interference in short-term memory

\textbf{Marsja}, E., Neely, Ljungberg, K.J (Under revision). Deviance
distraction in the auditory and tactile modalities after repeated
exposure: differential aspects of tactile and auditory deviants

\subsection{Current research
collaborations}\label{current-research-collaborations}

\hrule

\begin{itemize}
\tightlist
\item
  Associate Professor Jessica K. Ljungberg, Department of Psychology,
  Umeå University, Umeå, Sweden
\item
  Associate Professor Gregory Neely, Department of Psychology, Umeå
  University, Umeå, Sweden
\item
  Dr.~Patrik Hansson, Department of Psychology, Umeå University, Umeå,
  Sweden
\item
  Dr.~John E. Marsh, School of Psychology, University of Central
  Lancashire, UK
\item
  Professor Patrik Sörqkvist, Faculty of Engineering and Sustainable
  Development, Högskolan i Gävle, Gävle
\end{itemize}

\subsection{Conference Presentations}\label{conference-presentations}

\hrule

\textbf{Marsja}, E., Neely G., Ma, Lichen., Ljungberg K.J., (2015,
August). Cross-modality matches of intensity and attention capture
dimensions of auditory and vibrotactile stimuli. Fechner Day 2015. The
31st Annual Meeting of the International Society for Psychophysics,
Québec, CA. \textbf{Poster}.

\textbf{Marsja}, E., Neely, G., Parmentier, F.B.R., Ljungberg, K.J.,
(2014, October) Deviance Distraction Is Contingent on Stimuli Being
Presented within the Same Modality. Psychonomic Society's 55th Annual
Meeting. Long Beach, CA, USA. \textbf{Poster}.

Ljungberg, K.J., Parmentier, F.B.R., \textbf{Marsja}, E., Neely, G.,
Jones, D., (2014, January). Any Tom, Dick, or Harry will do: Hearing
one's own name distracts no more than any other in a cross-modal oddball
task. Experimental Psychology Society Meeting. London, UK.
\textbf{Poster}.

\textbf{Marsja}, E., Neely, G., Parmentier, F.B.R., Ljungberg, K.J.,
(2013, October). Maintenance of the distractive effect of deviating
vibrotactile stimuli in a cross-modal oddball paradigm. the 29th Annual
meeting of the International Society of Psychophysics, Freiburg, DE.
\textbf{Poster}.

\subsection{Funding and Grants}\label{funding-and-grants}

\hrule

\textbf{10 000 SEK} JC Kempes minnesfond for the project \emph{Is
everyday distractibility related to attention capture by vibrating
deviants?} (2014).

\textbf{9 000 SEK} from Knut och Alice Wallenbergs Stiftelse for
participating in the conferences Psychonomic Society's 55th Annual
Meeting, Long Beach, USA, 20-23 November and APCAM, Long Beach, USA, 20
November, 2014.

\textbf{6000 SEK} from the Department of Psychology for participating
the conference Fechner Day 2013 (the 29th Annual Meeting of the
International Society for Psychophysics) 21-25 October, Freiburg i.Br.,
Germany, 2013

\subsection{Teaching responsibilities}\label{teaching-responsibilities}

\subsubsection{Past \& current areas at undergraduate
level}\label{past-current-areas-at-undergraduate-level}

\hrule

Primarily in:

\begin{itemize}
\tightlist
\item
  Attention
\item
  Perception
\item
  Applied Cognitive Psychology
\item
  Cognitive Psychology
\item
  Research Methods
\item
  Project work
\end{itemize}

Lectures, Seminars, Lab demonstrations, Group project (both involving
empirical projects and more applied projects) and Supervision have been
given at the Department of Psychology Umeå University, Sweden. I have so
far taught 603 clock hours. See Table for an oeverview of teaching
topics. \blandscape

\begin{table}[!htbp] \centering    \caption{Teaching responsibilities - an overview of type of teaching, hours, etc.}    \label{}  \small  \begin{tabular}{@{\extracolsep{5pt}} lp{4cm}p{4cm}lp{5cm}ll} \\[-1.8ex]\hline  \hline \\[-1.8ex]  Period & Subject & Type & Clock Hours  & Course/Program & Level & Language \\  \hline \\[-1.8ex]  VT 2014 & Cognitive Psychology & Supervision of project work & 20 & Introduction to Psychology & Undergraduate & Swedish \\    & Research Methods & Lectures & 60 & Developmental Psychology and Personality Theory , Psychology Program & Undergraduate & Swedish \\    &  &  &   &  &  &  \\  HT 2014 & Cognitive Psychology & Supervision of project work & 20 & Introduction to Psychology & Undergraduate & Swedish \\    & Psychology & Supervision & 20 & Master Thesis 30 ECTS, Psychology Program & Graduate & Swedish \\    & Cognitive Science & Supervision of project work & 40 & Project in Cognitive Science, Bachelor's Program in Cognitive Science & Undergraduate & Swedish \\    &  &  &   &  &  &  \\  VT 2015 & Cognitive Psychology & Supervising projects & 24 & Introduction to Psychology & Undergraduate & Swedish \\    & Cognitive Psychology,  Attention, Perception, Memory & Lectures, seminars, supervision of projects,  & 20 & Basic psychology and sport psychology & Undergraduate & Swedish \\    & Cognitive Psychology, Psychophysics, Attention & Supervision & 15 & Master Thesis 15 ECTS, Master's Program in Cognitive Science & Graduate & English \\    & Cognitive Science & Supervision of project work & 110 & Project in Cognitive Science, Bachelor's Program in Cognitive Science & Undergraduate & Swedish \\    &  &  &   &  &  &  \\  HT 2015 & Cognitive Psychology & Supervising projects & 24 & Cognitive Psychology, Psychology Program & Undergraduate & Swedish \\    & Research Methods & Seminars in Qualitative research methods & 20 & Psychology of Organizations, Environment and Work, Psychology Program & Undergraduate & Swedish \\    &  &  &   &  &  &  \\  VT 2016 & Cognitive Psychology & Lab demonstrations & 30 & 2:1 Cognition, Psychology Program & Undergraduate & English \\    & Applied Cognitive Psychology,  Auditory Cognition, Attention & Lectures, seminars, supervision of projects, examination & 60 & Applied Cognitive Psychology, Bachelor's Program in Cognitive Science & Undergraduate & Swedish \\    & Research Methods & SPSS laboration, seminars, examination & 30 & Psychology of Organizations, Environment and Work, Psychology Program & Undergraduate & Swedish \\    & Cognitive Science & Supervision of project work & 110 & Project in Cognitive Science, Bachelor's Program in Cognitive Science & Undergraduate & Swedish \\  \hline \\[-1.8ex]  \multicolumn{7}{l}{ Total of clock hours: 603} \\  \end{tabular}  \end{table}

\elandscape

\subsubsection{Supervision of master
students}\label{supervision-of-master-students}

\hrule

Ma, L. (2015). Department of Psychology, Umeå University. ``Cross-Modal
Matching of Distractibility in Auditory and Tactile Stimuli''. Master
Thesis in Cognitive Science (15 ECTS)

Blide, M. (2014). Department of Psychology, Umeå University. ``Att orka
lämna ett misshandelsförhållande: Anknytningens beydelse (To cope
leaving abusive relationships: The importance of attachment). Master
Thesis in Clinical Psychology (30 ECTS)

\subsection{Additional skills}\label{additional-skills}

\hrule

\begin{itemize}
\tightlist
\item
  Good programming skills in Python (v2.x) and R
\item
  Good skills in Microsoft Word and Excel
\item
  Basic skills in Markdown (e.g., RMarkdown) and LaTex
\item
  Basic programming skills in Visual Basic, E-basic (E-prime), and
  MATLAB
\item
  Basic skills in statistical software such as SPSS and R
\item
  Programming and performing experiments using both E-prime, MATLAB, and
  Python (i.e., PsychoPy)
\end{itemize}

\noindent\makebox[\linewidth]{\rule{\textwidth}{0.4pt}}

\end{document}
