\documentclass[]{article}
\usepackage{lscape}

\usepackage{titling}
\usepackage{caption}
\usepackage{tabularx}
\usepackage{tikz}
\usepackage{xcolor}
\usepackage{fancyhdr}
\usepackage{lastpage}
\usepackage{lscape}
\usepackage{titlesec}
\usepackage[scaled=.90]{helvet}% Helvetica, served as a model for arial
\pagestyle{fancy}
\fancyhf{}
\renewcommand{\headrulewidth}{0pt}
\renewcommand{\footrulewidth}{0.4pt}
\fancyfoot[CO,CE]{CV - Erik Marsja - Page \thepage \hspace{1pt} of \pageref{LastPage}}
\fancypagestyle{plain}{\pagestyle{fancy}}

% Font awesome
\usepackage{fontawesome}

\usepackage[T1]{fontenc}
\usepackage[utf8]{inputenc}
\usepackage[swedish]{babel}

% For tables


% Tightlist

\providecommand{\tightlist}{%
  \setlength{\itemsep}{0pt}\setlength{\parskip}{0pt}}

% URLS
\usepackage[hidelinks]{hyperref}
\usepackage{float} % here for H placement parameter

% Appendix header style

\fancypagestyle{style2}{
\fancyhf{}
\fancyhead[C]{Appendix 1}
}


% For Changing the margins (Education)

\def\changemargin#1#2{\list{}{\rightmargin#2\leftmargin#1}\item[]}
\let\endchangemargin=\endlist 

%For displaying the table in landscape format
\usepackage[absolute]{textpos}

\fancypagestyle{lscape}{% 
\fancyhf{} % clear all header and footer fields 
\fancyfoot[LE]{%
\begin{textblock}{20}(1,5){\rotatebox{90}{\leftmark}}\end{textblock}
\begin{textblock}{1}(13,10.5){\rotatebox{90}{\thepage}}\end{textblock}}
\fancyfoot[LO] {%
\begin{textblock}{1}(13,10.5){\rotatebox{90}{\thepage}}\end{textblock}
\begin{textblock}{20}(1,13.25){\rotatebox{90}{\rightmark}}\end{textblock}}
\renewcommand{\headrulewidth}{0pt} 
\renewcommand{\footrulewidth}{0pt}}

\setlength{\TPHorizModule}{1cm}
\setlength{\TPVertModule}{1cm}


\newcommand\secbar {
    \tikz[baseline, trim left=3.2cm] 
    {
        \fill [white] (3cm,0.1ex) rectangle +(0.2cm,1.1ex);
        \draw [gray!95, fill=gray!80] (0cm,0.1ex) rectangle (3cm,1.1ex);        
    }
}
\newcommand\subsecbar {
    \tikz[baseline, trim left=0.15cm] 
    {
        \fill [white] (2cm,0.1ex) rectangle +(0.2cm,1.1ex);
        \fill [blue!40] (0cm,0.1ex) rectangle (2cm,1.1ex);      
    }
}

\newcommand\subsubsecbar {
    \tikz[baseline, trim left=0.15cm] 
    {
        \fill [white] (1cm,0.1ex) rectangle +(0.2cm,1.1ex);
        \fill [blue!40] (0cm,0.1ex) rectangle (2cm,1.1ex);      
    }
}

\titleformat{\section}{\large}{}{0cm}{\secbar}
\titleformat{\subsection}{\large}{}{0cm}{\normalfont\sffamily\Large\bfseries\subsecbar}
\titleformat{\subsubsection}{}{}{0cm}{\normalfont\sffamily\large\bfseries}

% No first line paragraph indent
\usepackage{parskip}
\usepackage{enumitem}


\titlespacing\section{0pt}{12pt plus 4pt minus 2pt}{4pt plus 2pt minus 2pt}
\titlespacing\subsection{0pt}{12pt plus 4pt minus 2pt}{4pt plus 2pt minus 2pt}
\titlespacing\subsubsection{0pt}{12pt plus 4pt minus 2pt}{4pt plus 2pt minus 2pt}

\newcolumntype{Y}{>{\centering\arraybackslash}X}

\begin{document}

\centerline{\huge \textbf{Erik Marsja} | \textcolor{darkgray}{Curriculum Vitae}}

\vspace{2 mm}

\hrule

\begin{table}[h]
\centering
\begin{tabularx}{\textwidth}{@{}lYl@{}}
\multicolumn{3}{c}{Kandidatvägen 1, SE-907 33 Umeå, Sweden} \\\hline
 \faPhone \hspace{1 mm}  +4670-36 33 662  \hspace{1 mm}  &  & \faEnvelopeO \hspace{1 mm} \href{mailto:}{\tt \href{mailto:erik@marsja.se}{\nolinkurl{erik@marsja.se}}} \hspace{1 mm}  \\
 \faGlobe \hspace{1 mm} \href{http://marsja.se}{\tt marsja.se}   &  & \faGithub \hspace{1 mm} \href{http://github.com/marsja}{\tt marsja} \hspace{1 mm}  
 \\\hline
\end{tabularx}
\end{table}

\subsection{Education}\label{education}

2017 \hspace{1.5cm}\textbf{Ph.D. in Psychology} \hrule

\begin{changemargin}{2.3cm}{2.4cm}

Department of Psychology, Umeå University.

\textbf{Thesis title}: Attention capture by sudden and unexpected changes: a multisensory perspective. 
Available from the DiVA Database: http://umu.diva-portal.org/smash/record.jsf?pid=diva2%3A1156775)

\textbf{Supervisors}: Associate Professor Jessica K. Ljungberg, Professor Gregory Neely, \& Dr. Patrik Hansson

\end{changemargin}

2012 \hspace{1.5cm}\textbf{M.Sc. in Cognitive Science} \hrule

\begin{changemargin}{2.3cm}{2.4cm}


\textbf{Thesis title}: Attention Capture: The Impact of Change in Spatial Sound Source on Behavior. 
    
\textbf{Supervisor}: Associate Professor Jessica K. Ljungberg
    
\end{changemargin}

2011 \hspace{1.5cm}\textbf{B.Sc. in Cognitive Science} \hrule

\begin{changemargin}{2.3cm}{2.4cm}


Department of Psychology, Umeå University.

\textbf{Thesis title}:  Attention Capture: Studying the Distracting Effect of One’s Own Name. 
Available from DiVA Database: http://urn.kb.se/resolve?urn=urn:nbn:se:umu:diva-46607.
    
\textbf{Supervisor}: Associate Professor Jessica K. Ljungberg

\end{changemargin}

\subsection{Publications}\label{publications}

\subsubsection{International peer-reviewed
journals}\label{international-peer-reviewed-journals}

\textbf{Marsja}, E., Neely, G., \& Ljungberg, J. K. (2018).
Investigating Deviance Distraction and the Impact of the Modality of the
To-Be-Ignored Stimuli. Experimental Psychology, 65(2), 61--70.
\url{http://doi.org/10.1027/1618-3169/a000390}

Ljungberg, K. J., Parmentier, F. B. R., Jones, D. M., \textbf{Marsja},
E., \& Neely, G. (2014). ``What's in a name?'' ``No more than when it's
mine own''. Evidence from auditory oddball distraction. Acta
Psychologica, 150C, 161--166. \url{doi:10.1016/j.actpsy.2014.05.009}.

\subsubsection{Publications In Preparation/Submitted/Under
Revision}\label{publications-in-preparationsubmittedunder-revision}

Rosa, E., \textbf{Marsja}, E., Ljungberg K.J. (Manuscript in
Preparation). Tactile warnings constitute an efficient alarm during high
mental workload in simulated flight tasks

Marsh, J.E., Vachon, F., Sörqvist, P., \textbf{Marsja}, E., Röer J.P.,
\& Ljungberg, K.J. (Under Revision). Irrelevant vibro-tactile stimuli
produce a changing-state effect: Implications for theories of
interference in short-term memory.

\textbf{Marsja}, E., Neely, G., \& Ljungberg, K.J (Manuscript in
Preparation). Deviance distraction in the auditory and tactile
modalities after repeated exposure: differential aspects of tactile and
auditory deviants.

\textbf{Marsja}, E., Marsh, J.E., Hansson, P., \& Neely, G. (Manuscript
in Preparation). Examining the Role of Spatial Changes in Bimodal and
Uni-Modal To-Be-Ignored Stimuli and How They Affect Short-Term Memory
Processes.

\subsection{Conference Presentations}\label{conference-presentations}

\textbf{Marsja}, E., Marsh, J.E., Neely G., Hansson P., Ljungberg K.J.,
(2017, April). Domain-generality or domain-specificity of the short-term
memory: insights from a multisensory distraction paradigm. Re-thinking
the Senses Spring School, Dubrovnik, Croatia. \textbf{Poster}.

\textbf{Marsja}, E., Marsh, J.E., Neely G., Hansson P., Ljungberg K.J.,
(2016, September). Do Spatial Changes in Sounds and Vibrations Affect
Visuo-spatial and Verbal Short-Term Memory? Attention and Control:
Insights from Distraction, Workshop, University of Central Lancashire,
Preston, UK. \textbf{Invited Talk}.

\textbf{Marsja}, E., Marsh, J.E., Neely G., Hansson P., Ljungberg K.J.,
(2016, June). Spatial Change in Multisensory Distractors Impact on
Verbal and Spatial Short Term Memory. International Multisensory
Research Forum 17\textsuperscript{th} annual meeting, Suzhou, CHN.
\textbf{Oral presentation}.

\textbf{Marsja}, E., Neely G., Ma, L., Ljungberg K.J., (2015, August).
Cross-modality matches of intensity and attention capture dimensions of
auditory and vibrotactile stimuli. Fechner Day 2015. The
31\textsuperscript{st} Annual Meeting of the International Society for
Psychophysics, Québec, CA. \textbf{Poster}.

\textbf{Marsja}, E., Neely, G., Parmentier, F.B.R., Ljungberg, K.J.,
(2014, October) Deviance Distraction Is Contingent on Stimuli Being
Presented within the Same Modality. Psychonomic Society's
55\textsuperscript{th} Annual Meeting. Long Beach, CA, USA.
\textbf{Poster}.

Ljungberg, K.J., Parmentier, F.B.R., \textbf{Marsja}, E., Neely, G.,
Jones, D., (2014, January). Any Tom, Dick, or Harry will do: Hearing
one's own name distracts no more than any other in a cross-modal oddball
task. Experimental Psychology Society Meeting. London, UK.
\textbf{Poster}.

\textbf{Marsja}, E., Neely, G., Parmentier, F.B.R., Ljungberg, K.J.,
(2013, October). Maintenance of the distractive effect of deviating
vibrotactile stimuli in a cross-modal oddball paradigm. the
29\textsuperscript{th} Annual meeting of the International Society of
Psychophysics, Freiburg, DE. \textbf{Poster}.

\subsection{Funding and Grants}\label{funding-and-grants}

\textbf{6000 SEK} from the Department of Psychology, Umeå University,
for participating in the Re-thinking the Senses Spring School,
Dubrovnik, Croatia, 2017.

\textbf{15000 SEK} from Lars Hiertas Minnesfond for the project \emph{Är
korttidsminnet domän-generellt eller domän-specifikt? (Is short-term
memory domain-general or domain specific?)}, 2016.

\textbf{12 000 SEK} from the Faculty of Social sciences, Umeå
University, for participating in the workshop Attention and Control:
Insights from Distraction and visiting a researcher at the University of
Central Lancashire, Preston, UK, 2016.

\textbf{8000 SEK} from the Department of Psychology, Umeå University,
for participating the conference 17\textsuperscript{th} International
Multisensory Research Forum 15-18 June, Souzou, China, 2016.

\textbf{10 000 SEK} from JC Kempes minnesfond for the project \emph{Is
everyday distractibility related to attention capture by vibrating
deviants?}, 2014.

\textbf{9 000 SEK} from Knut och Alice Wallenbergs Stiftelse for
participating in the conferences Psychonomic Society's
55\textsuperscript{th} Annual Meeting, 20-23 November, and APCAM, 20
November, Long Beach, USA, 2014.

\textbf{6000 SEK} from the Department of Psychology for participating
the conference Fechner Day 2013 (the 29\textsuperscript{th} Annual
Meeting of the International Society for Psychophysics) 21-25 October,
Freiburg i.Br., Germany, 2013.

\subsection{Teaching Experience}\label{teaching-experience}

\subsubsection{Past \& current areas at undergraduate level at Umeå
University}\label{past-current-areas-at-undergraduate-level-at-umea-university}

Primarily in:

\begin{itemize}
\tightlist
\item
  Attention
\item
  Perception
\item
  Applied Cognitive Psychology
\item
  Cognitive Psychology
\item
  Research Methods
\end{itemize}

Lectures, seminars, lab demonstrations, supervision of group project
(both involving empirical and applied projects) and supervision of
thesis' have been given at the Department of Psychology Umeå University,
Sweden. I have so far taught 950 clock hours.

\subsubsection{Supervision of Master students (graduate
level)}\label{supervision-of-master-students-graduate-level}

Ma, L. (2015). Department of Psychology, Umeå University. ``Cross-Modal
Matching of Distractibility in Auditory and Tactile Stimuli''. Master
Thesis in Cognitive Science (15 ECTS).

Blide, M. (2014). Department of Psychology, Umeå University. ``Att orka
lämna ett misshandelsförhållande: Anknytningens beydelse (To cope
leaving abusive relationships: The importance of attachment). Master
Thesis in Clinical Psychology (30 ECTS).

\subsection{Additional skills}\label{additional-skills}

\begin{itemize}
\tightlist
\item
  Good skills in Microsoft Word and Excel
\item
  Good skills in statistical software such as SPSS and R
\item
  Good programming skills in Python (v2.7.x) and R
\item
  Basic programming skills in Visual Basic, E-basic (E-prime), and
  MATLAB
\item
  Basic skills in Markdown (e.g., RMarkdown) and \LaTeX
\item
  Programming and performing experiments using both E-prime, MATLAB, and
  Python (i.e., PsychoPy, OpenSesame, \& Expyriment)
\end{itemize}

\subsubsection{Responsibilities}\label{responsibilities}

\begin{itemize}
\tightlist
\item
  Elected chairperson of the Ph.D.~students Council of the Department of
  Psychology, Umeå University.
\end{itemize}

\end{document}