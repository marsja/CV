\documentclass[]{article}
\usepackage{pdflscape}

\usepackage{everypage}

\newcommand{\Lpagenumber}{\ifdim\textwidth=\linewidth\else\bgroup
  \dimendef\margin=0 %use \margin instead of \dimen0
  \ifodd\value{page}\margin=\oddsidemargin
  \else\margin=\evensidemargin
  \fi
  \raisebox{\dimexpr -\topmargin-\headheight-\headsep-0.5\linewidth}[0pt][0pt]{%
    \rlap{\hspace{\dimexpr \margin+\textheight+\footskip}%
    \llap{\rotatebox{90}{ CV  - Erik Marsja 
- Page \thepage of \pageref{LastPage}}}}}%
\egroup\fi}
\AddEverypageHook{\Lpagenumber}%

\usepackage{adjustbox}
\usepackage{titling}
\usepackage{caption}
\usepackage{tabularx}
\usepackage{tikz}
\usepackage{xcolor}
\usepackage{fancyhdr}
\usepackage{lastpage}
\usepackage{titlesec}
\usepackage[scaled=.90]{helvet}% Helvetica, served as a model for arial

\pagestyle{fancy}
\fancyhf{}
\renewcommand{\headrulewidth}{0pt}
\renewcommand{\footrulewidth}{0.0pt}
\fancyfoot[CO,CE]{  CV -   Erik Marsja -  DNR: AN 2.2.1-1774-18 -  \thepage\hspace{1pt}(\pageref{LastPage})}
\fancypagestyle{plain}{\pagestyle{fancy}}

\usepackage[margin=1.2in]{geometry}

% Font awesome
\usepackage{fontawesome}

\usepackage[T1]{fontenc}
\usepackage[utf8]{inputenc}
\usepackage[Swedish, English]{babel}

% For tables


% Tightlist

\providecommand{\tightlist}{%
  \setlength{\itemsep}{0pt}\setlength{\parskip}{0pt}}

% URLS
\usepackage[hidelinks]{hyperref}
\usepackage{breakurl}
\usepackage{float} % here for H placement parameter

% Appendix header style

\fancypagestyle{style2}{
\fancyhf{}
\fancyhead[C]{Appendix 1}
}


% For Changing the margins (Education)

\def\changemargin#1#2{\list{}{\rightmargin#2\leftmargin#1}\item[]}
\let\endchangemargin=\endlist 

%For displaying the table in landscape format
\usepackage[absolute]{textpos}

\fancypagestyle{lscape}{% 
\fancyhf{} % clear all header and footer fields 
\fancyfoot[LE]{%
\begin{textblock}{20}(1,5){\rotatebox{90}{\leftmark}}\end{textblock}
\begin{textblock}{1}(13,10.5){\rotatebox{90}{\thepage}}\end{textblock}}
\fancyfoot[LO] {%
\begin{textblock}{1}(13,10.5){\rotatebox{90}{\thepage}}\end{textblock}
\begin{textblock}{20}(1,13.25){\rotatebox{90}{\rightmark}}\end{textblock}}
\renewcommand{\headrulewidth}{0pt} 
\renewcommand{\footrulewidth}{0pt}}

\setlength{\TPHorizModule}{1cm}
\setlength{\TPVertModule}{1cm}

% Appendix
\fancypagestyle{style2}{
\fancyhf{}
\fancyhead[C]{Appendix 1}
\renewcommand{\headrulewidth}{1pt}
}



\newcommand\secbar {
    \tikz[baseline, trim left=3.2cm] 
    {
        \fill [white] (3cm,0.1ex) rectangle +(0.2cm,1.1ex);
        \draw [gray!95, fill=gray!80] (0cm,0.1ex) rectangle (3cm,1.1ex);        
    }
}
\newcommand\subsecbar {
    \tikz[baseline, trim left=0.15cm] 
    {
        \fill [white] (2cm,0.1ex) rectangle +(0.2cm,1.1ex);
        \fill [blue!40] (0cm,0.1ex) rectangle (2cm,1.1ex);      
    }
}

\newcommand\subsubsecbar {
    \tikz[baseline, trim left=0.15cm] 
    {
        \fill [white] (1cm,0.1ex) rectangle +(0.2cm,1.1ex);
        \fill [blue!40] (0cm,0.1ex) rectangle (2cm,1.1ex);      
    }
}

\titleformat{\section}{\large}{}{0cm}{\secbar}
\titleformat{\subsection}{\large}{}{0cm}{\normalfont\sffamily\Large\bfseries\subsecbar}
\titleformat{\subsubsection}{}{}{0cm}{\normalfont\sffamily\large\bfseries}

% No first line paragraph indent
\usepackage{parskip}
\usepackage{enumitem}


\titlespacing\section{0pt}{12pt plus 4pt minus 2pt}{4pt plus 2pt minus 2pt}
\titlespacing\subsection{0pt}{12pt plus 4pt minus 2pt}{4pt plus 2pt minus 2pt}
\titlespacing\subsubsection{0pt}{12pt plus 4pt minus 2pt}{4pt plus 2pt minus 2pt}

\newcolumntype{Y}{>{\centering\arraybackslash}X}

\begin{document}

\centerline{\huge \textbf{Erik Marsja} | \textcolor{darkgray}{Curriculum Vitae}}

\vspace{2 mm}

\hrule

\begin{table}[h]
\centering
\begin{tabularx}{\textwidth}{@{}lYl@{}}
\textbf{Home Address}: & &  \textbf{DOB:} 
\\Kandidatvägen 1, SE-907 33 Umeå, Sweden & &  19810526 
\\\\

 \faPhone \hspace{1 mm}  +4670-36 33 662  \hspace{1 mm}  &  & \faEnvelopeO \hspace{1 mm} \href{mailto:}{\tt \href{mailto:erik@marsja.se}{\nolinkurl{erik@marsja.se}}} \hspace{1 mm}  \\
 \faGlobe \hspace{1 mm} \href{http://www.marsja.se}{\tt www.marsja.se}   &  & \faGithub \hspace{1 mm} \href{http://github.com/marsja}{\tt marsja} \hspace{1 mm}  \\
 \multicolumn{3}{c}{\emph{Languages: }Swedish, English}
 \\\hline
\end{tabularx}
\end{table}

\subsection{Education}\label{education}

2017 \hspace{1.5cm}\textbf{Ph.D. in Psychology\strut} \hrule

\begin{changemargin}{2.3cm}{2.4cm}

Department of Psychology, Umeå University.

\textbf{Thesis title}: Attention capture by sudden and unexpected changes: a multisensory perspective. 

Available from the DiVA Database: \sloppy http://umu.diva-portal.org/smash/record.jsf?pid=diva2%3A1156775

\textbf{Supervisors}: Associate Professor Jessica K. Ljungberg, Professor Gregory Neely, \& Dr. Patrik Hansson
\end{changemargin}

2012 \hspace{1.5cm}\textbf{M.Sc. in Cognitive Science\strut}

\hrule

\begin{changemargin}{2.3cm}{2.4cm}

Department of Psychology, Umeå University.

\textbf{Thesis title}: Attention Capture: The Impact of Change in Spatial Sound Source on Behavior. 
    
\textbf{Supervisor}: Associate Professor Jessica K. Ljungberg
\end{changemargin}

2011 \hspace{1.5cm}\textbf{B.Sc. in Cognitive Science\strut} \hrule

\begin{changemargin}{2.3cm}{2.4cm}

Department of Psychology, Umeå University.

\textbf{Thesis title}:  Attention Capture: Studying the Distracting Effect of One’s Own Name.

Available from DiVA Database: \sloppy http://urn.kb.se/resolve?urn=urn:nbn:se:umu:diva-46607.
    
\textbf{Supervisor}: Associate Professor Jessica K. Ljungberg
\end{changemargin}

\subsection{Employment}\label{employment}

Aug 2018 -

Oct 2018\hspace{0.75cm}\textbf{Postdoctoral Research Assistant\strut}

\hrule

\begin{changemargin}{2.3cm}{2.4cm}

Department of Psychology, Umeå University.

\textbf{Project}: How does humans perception of safety differ in a broken communication chain compared to an unbroken communication chain? 

Literature study commissioned by SOS Alarm, an emergency agency (e.g., public-safety answering point), with the results dissiminating in a report.

\end{changemargin}

Oct 2012 -

June
2018\hspace{0.75cm}\textbf{PhD Student and Teaching Assistant\strut}

\hrule

\begin{changemargin}{2.3cm}{2.4cm}

Department of Psychology, Umeå University.

Planning of studies, programming of experiments, literature search, data analysis, scientific communication, and many more. See the section "Teaching responsibilities" for an overview of my pedagogical experience.

\end{changemargin}

June 2011\hspace{0.75cm}\textbf{Undergrad. Research Assistant\strut}

\hrule

\begin{changemargin}{2.3cm}{2.4cm}

Department of Psychology, Umeå University.

Recruitment of participants and data collection.

\end{changemargin}

\subsection{Publications}\label{publications}

\subsubsection{International peer-reviewed
journals}\label{international-peer-reviewed-journals}

\textbf{Marsja}, E., Neely, G., \& Ljungberg, J. K. (2018).
Investigating Deviance Distraction and the Impact of the Modality of the
To-Be-Ignored Stimuli. \emph{Experimental Psychology, 65(2)}, 61--70.
\url{http://doi.org/10.1027/1618-3169/a000390}

Ljungberg, K. J., Parmentier, F. B. R., Jones, D. M., \textbf{Marsja},
E., \& Neely, G. (2014). ``What's in a name?'' ``No more than when it's
mine own''. Evidence from auditory oddball distraction. \emph{Acta
Psychologica, 150C}, 161--166. \url{doi:10.1016/j.actpsy.2014.05.009}.

\subsubsection{Publications In Preparation/Submitted/Under
Revision}\label{publications-in-preparationsubmittedunder-revision}

\textbf{Marsja}, E. Elbe, P., Sörman, D. (Manuscript in Preparation).
Examining the Factor Structure of the Swedish Translation of the
Attentional Control Scale and it's relation to objective measures of
attention.

Rosa, E., \textbf{Marsja}, E., Ljungberg, K.J. (Manuscript in
Preparation). Tactile warnings constitute an efficient alarm during high
mental workload in simulated flight tasks

Marsh, J.E., Vachon, F., Sörqvist, P., \textbf{Marsja}, E., Röer J.P.,
\& Ljungberg, K.J. (Under Revision). Irrelevant vibro-tactile stimuli
produce a changing-state effect: Implications for theories of
interference in short-term memory.

\textbf{Marsja}, E., Neely, G., \& Ljungberg, K.J (Under Review).
Deviance distraction in the auditory and tactile modalities after
repeated exposure: differential aspects of tactile and auditory
deviants.

\textbf{Marsja}, E., Marsh, J.E., Hansson, P., \& Neely, G. (Submitted).
Examining the Role of Spatial Changes in Bimodal and Uni-Modal
To-Be-Ignored Stimuli and How They Affect Short-Term Memory Processes.

\subsection{Conference Presentations}\label{conference-presentations}

\textbf{Marsja}, E., Marsh, J.E., Neely G., Hansson P., Ljungberg K.J.,
(2017, April). Domain-generality or domain-specificity of the short-term
memory: insights from a multisensory distraction paradigm. Re-thinking
the Senses Spring School, Dubrovnik, Croatia. \textbf{Poster}.

\textbf{Marsja}, E., Marsh, J.E., Neely G., Hansson P., Ljungberg K.J.,
(2016, September). Do Spatial Changes in Sounds and Vibrations Affect
Visuo-spatial and Verbal Short-Term Memory? Attention and Control:
Insights from Distraction, Workshop, University of Central Lancashire,
Preston, UK. \textbf{Invited Talk}.

\textbf{Marsja}, E., Marsh, J.E., Neely G., Hansson P., Ljungberg K.J.,
(2016, June). Spatial Change in Multisensory Distractors Impact on
Verbal and Spatial Short Term Memory. International Multisensory
Research Forum 17\textsuperscript{th} annual meeting, Suzhou, CHN.
\textbf{Oral presentation}.

\textbf{Marsja}, E., Neely G., Ma, L., Ljungberg K.J., (2015, August).
Cross-modality matches of intensity and attention capture dimensions of
auditory and vibrotactile stimuli. Fechner Day 2015. The
31\textsuperscript{st} Annual Meeting of the International Society for
Psychophysics, Québec, CA. \textbf{Poster}.

\textbf{Marsja}, E., Neely, G., Parmentier, F.B.R., Ljungberg, K.J.,
(2014, October) Deviance Distraction Is Contingent on Stimuli Being
Presented within the Same Modality. Psychonomic Society's
55\textsuperscript{th} Annual Meeting. Long Beach, CA, USA.
\textbf{Poster}.

Ljungberg, K.J., Parmentier, F.B.R., \textbf{Marsja}, E., Neely, G.,
Jones, D., (2014, January). Any Tom, Dick, or Harry will do: Hearing
one's own name distracts no more than any other in a cross-modal oddball
task. Experimental Psychology Society Meeting. London, UK.
\textbf{Poster}.

\textbf{Marsja}, E., Neely, G., Parmentier, F.B.R., Ljungberg, K.J.,
(2013, October). Maintenance of the distractive effect of deviating
vibrotactile stimuli in a cross-modal oddball paradigm. The
29\textsuperscript{th} Annual meeting of the International Society of
Psychophysics, Freiburg, DE. \textbf{Poster}.

\subsection{Popular Science and Education-Related
Articles}\label{popular-science-and-education-related-articles}

\textbf{Marsja}, E. (2016, July). Python Programming in Psychology --
From Data Collection to Analysis. \emph{JEPS Bulletin - The Official
Blog of the Journal of European Psychology Students}. \textbf{Invited
Blog Post}. Retrieved from \sloppy
\url{http://blog.efpsa.org/2016/07/12/python-programming-in-psychology-from-data-collection-to-analysis/}

\subsection{Funding and Grants}\label{funding-and-grants}

\textbf{6000 SEK} from the Department of Psychology, Umeå University,
for participating in the Re-thinking the Senses Spring School,
Dubrovnik, Croatia, 2017.

\textbf{15000 SEK} from Lars Hiertas Minnesfond for the project \emph{Är
korttidsminnet domän-generellt eller domän-specifikt? (Is short-term
memory domain-general or domain specific?)}, 2016.

\textbf{12 000 SEK} from the Faculty of Social sciences, Umeå
University, for participating in the workshop Attention and Control:
Insights from Distraction, and visiting a researcher at the University
of Central Lancashire, Preston, UK, 2016.

\textbf{8000 SEK} from the Department of Psychology, Umeå University,
for participating the conference 17\textsuperscript{th} International
Multisensory Research Forum 15-18 June, Souzou, China, 2016.

\textbf{10 000 SEK} from JC Kempes minnesfond for the project \emph{Is
everyday distractibility related to attention capture by vibrating
deviants?}, 2014.

\textbf{9 000 SEK} from Knut och Alice Wallenbergs Stiftelse for
participating in the conferences Psychonomic Society's
55\textsuperscript{th} Annual Meeting, 20-23 November, and APCAM, 20
November, Long Beach, USA, 2014.

\textbf{6000 SEK} from the Department of Psychology for participating
the conference Fechner Day 2013 (the 29\textsuperscript{th} Annual
Meeting of the International Society for Psychophysics) 21-25 October,
Freiburg i.Br., Germany, 2013.

\subsection{Peer-reviewed for
Journals:}\label{peer-reviewed-for-journals}

\begin{itemize}
\tightlist
\item
  Journal of Cognitive Psychology
\end{itemize}

Information about my responsibilities as a peer-reviewer can be found in
my \href{https://www.publons.com/a/1517052/}{Publons profile}.

\subsection{Skills \& Training}\label{skills-training}

\subsubsection{Training}\label{training}

2017 Spring School \emph{Re-Thinking the Senses}, Inter-University
Centre, Dubrovnik, Croatia

\subsubsection{Teaching}\label{teaching}

Primarily in:

\begin{itemize}
\tightlist
\item
  Attention
\item
  Perception
\item
  Applied Cognitive Psychology
\item
  Cognitive Psychology
\item
  Research Methods
\end{itemize}

Lectures, seminars, lab demonstrations, supervision of group project
(both involving empirical and applied projects) and supervision of
thesis' have been given at the Department of Psychology Umeå University,
Sweden. I have so far taught 950 clock hours. See Table 1 for an
overview of my teaching responsibilities, clock hours, and courses.

\subsubsection{Supervision of Master students (graduate
level)}\label{supervision-of-master-students-graduate-level}

Ma, L. (2015). Department of Psychology, Umeå University. ``Cross-Modal
Matching of Distractibility in Auditory and Tactile Stimuli''. Master
Thesis in Cognitive Science (15 ECTS).

Blide, M. (2014). Department of Psychology, Umeå University. ``Att orka
lämna ett misshandelsförhållande: Anknytningens beydelse (To cope
leaving abusive relationships: The importance of attachment). Master
Thesis in Clinical Psychology (30 ECTS).

\subsubsection{Teaching Workshops}\label{teaching-workshops}

Oct 2018\hspace{0.75cm}\textbf{R Workshop "Step-by-Step"\strut}

\hrule

\begin{changemargin}{2.15cm}{2.4cm}


Department of Psychology, Umeå University.

An introduction to R statistical programming language - presented to senior researchers with a focus on basic programming and the R environment.

\end{changemargin}

\subsubsection{Additional skills}\label{additional-skills}

\begin{itemize}
\tightlist
\item
  Extensive knowledge in statistical software such as SPSS, JASP, and R
  statistical programming environment
\item
  Strong scripting skills in Python (v2.7.x \& v3.x.x) and R
\item
  Substantial skills in programming and performing experiments using
  both E-prime and Python (i.e., PsychoPy, OpenSesame, \& Expyriment)
\item
  Good skills in Microsoft Word and Excel
\item
  Basic programming skills in Visual Basic, E-basic (E-prime), MATLAB,
  Bash, JavaScript, and PHP
\item
  Basic skills in Markdown (e.g., RMarkdown) and \LaTeX
\end{itemize}

\subsubsection{Responsibilities}\label{responsibilities}

\begin{itemize}
\tightlist
\item
  Elected chairperson of the Ph.D.~students Council of the Department of
  Psychology, Umeå University.
\end{itemize}

\begin{landscape}
\pagestyle{empty}


{\selectlanguage{english}

   \begin{table}[!htbp] \centering    \caption{Teaching responsibilities - an overview of type of teaching, hours, etc.}    \label{}  \tiny  \begin{tabular}{@{\extracolsep{5pt}} lp{4cm}p{4cm}lp{5cm}ll}  \\[-1.8ex]\hline  \hline \\[-1.8ex]  Period & Subject & Type & Clock Hours  & Course/Program & Level & Language \\  \hline \\[-1.8ex]  VT 2014 & Scientific communication, research methods, research ethics & Lectures, seminars, supervision & 60 & Psychological test and research methods/Statistical and Empirial Methods, the Clinical Psychologist program/the Bachelor's Program in Cognitive Science & UG & Swe \\    & Cognitive Psychology & Supervision of projects & 20 & Introduction to Psychology, the Clincial Psychologist Program & UG & Swe \\    &  &  &   &  &  &  \\  HT 2014 & Cognitive Psychology & Supervision of projects & 20 & Introduction to Psychology, the Clincial Psychologist Program & UG & Swe \\    & Clinical Psychology & Co-supervision of thesis project & 20 & Master Thesis in Psychology, 30 ECTS, the Clinical Psychologist Program & G & Eng \\    & Applied Cognitive Science & Supervision of projects & 40 & Project in Cognitive Science, the Bachelor's Program in Cognitive Science & UG & Swe \\    &  &  &   &  &  &  \\  VT 2015 & Cognitive Psychology & Supervision of projects & 24 & Introduction to Psychology, the Clincial Psychologist Program & UG & Swe \\    & Cognitive Psychology & Lectures, seminars, supervision of projects & 20 & Basic psychology and sport psychology, the Coaching Program & UG & Swe \\    & Cognitive Psychology & Supervision of thesis project & 15 & Master Thesis in Cognitive Science, 15 ECTS, the Master's Program in Cognitive Science & G & Eng \\    & Applied Cognitive Science & Supervision of projects & 110 & Project in Cognitive Science, the Bachelor's Program in Cognitive Science & UG & Swe \\    &  &  &   &  &  &  \\  HT 2015 & Cognitive Psychology & Supervision of projects & 24 & Cognitive Psychology, the Clinical Psychologist Program & UG & Swe \\    & Qualitative methods & Seminars & 20 & the Psychology of Organizations, Environment and Work, the Clinical Psychologist Program & UG & Swe \\    &  &  &   &  &  &  \\  VT 2016 & Cognitive Psychology & Lab demonstrations & 30 & 2:1 Cognition, the Clinical Psychologist Program & UG & Eng \\    & Cognitive Psychology: Attention & Lectures, seminars, supervision of projects & 60 & Applied Cognitive Psychology, the Bachelor's Program in Cognitive Science & UG & Swe \\    & Qualitative and Quantitative methods & Seminars, computer labs & 30 & the Psychology of Organizations, Environment and Work, the Clinical Psychologist Program & UG & Swe \\    & Applied Cognitive Science & Supervision of projects & 110 & Project in Cognitive Science, the Bachelor's Program in Cognitive Science & UG & Swe \\    &  &  &   &  &  &  \\  HT 2016 & Cognitive Psychology & Lab demonstrations & 30 & 2:1 Cognition, the Clinical Psychologist Program & UG & Eng \\    & Qualitative and Quantitative methods & Seminars, computer labs & 30 & the Psychology of Organizations, Environment and Work, the Clinical Psychologist Program & UG & Swe \\    & Cognitive Psychology: Perception, Attention, and Consciousness & Lectures & 24 & Perception, the Bachelor's Program in Cognitive Science & UG & Swe \\    & Clinical Psychology & Seminars & 9 & Master Thesis in Psychology, 30 ECTS, the Clinical Psychologist Program & G & Swe \\    &  &  &   &  &  &  \\  VT 2017 & Cognitive Psychology & Lab demonstrations & 30 & 2:1 Cognition, the Clinical Psychologist Program & UG & Eng \\    & Cognitive Psychology, Applied: Attention & Lectures, seminar & 20 & Applied Cognitive Psychology, the Bachelor's Program in Cognitive Science & UG & Swe \\    & Applied Cognitive Science & Supervision of projects & 40 & Project in Cognitive Science, the Bachelor's Program in Cognitive Science & UG & Swe \\    & Qualatitive and Quantative research methods & Seminars, computer labs & 30 & the Psychology of Organizations, Environment and Work, the Clinical Psychologist Program & UG & Swe \\    & Clinical Psychology & Seminars & 30 & Master Thesis in Psychology, 30 ECTS, the Clinical Psychologist Program & G & Swe \\    &  &  &   &  &  &  \\  HT 2017 & Cognitive Psychology & Lab demonstrations & 30 & 2:1 Cognition, the Clinical Psychologist Program & UG & Eng \\    & Cognitive Psychology: Attention and Consciousness & Lectures, seminars & 24 & Perception, the Bachelor's Program in Cognitive Science & UG & Swe \\    & Developmental Psychology & Seminar leader & 50 & Learning and teaching, pedagogy & UG & Swe \\  \hline \\[-1.8ex]  \multicolumn{7}{l}{ Total of clock hours: 950 , UG = Undergraduate, G = Graduate} \\  \end{tabular}  \end{table} 
}
\end{landscape}

\end{document}