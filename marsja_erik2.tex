\documentclass[]{article}
\usepackage{pdflscape}

\usepackage{everypage}

\newcommand{\Lpagenumber}{\ifdim\textwidth=\linewidth\else\bgroup
  \dimendef\margin=0 %use \margin instead of \dimen0
  \ifodd\value{page}\margin=\oddsidemargin
  \else\margin=\evensidemargin
  \fi
  \raisebox{\dimexpr -\topmargin-\headheight-\headsep-0.5\linewidth}[0pt][0pt]{%
    \rlap{\hspace{\dimexpr \margin+\textheight+\footskip}%
    \llap{\rotatebox{90}{ CV  - Erik Marsja 
- \thepage\hspace{1pt}(\pageref{LastPage})}}}}%
\egroup\fi}
\AddEverypageHook{\Lpagenumber}%

\usepackage{adjustbox}
\usepackage{titling}
\usepackage{caption}
\usepackage{tabularx}
\usepackage{tikz}
\usepackage{xcolor}
\usepackage{fancyhdr}
\usepackage{lastpage}
\usepackage{titlesec}
\usepackage[scaled=.90]{helvet}% Helvetica, served as a model for arial

% Added for new table:
\usepackage{booktabs}
\usepackage{threeparttablex}
\usepackage{longtable}

\pagestyle{fancy}
\fancyhf{}
\renewcommand{\headrulewidth}{0pt}
\renewcommand{\footrulewidth}{0.0pt}
\fancyfoot[CO,CE]{  CV -   Erik Marsja -  \thepage\hspace{1pt}(\pageref{LastPage})}
\fancypagestyle{plain}{\pagestyle{fancy}}

\usepackage[margin=1.2in]{geometry}

% Font awesome
\usepackage{fontawesome}

\usepackage[T1]{fontenc}
\usepackage[utf8]{inputenc}

% For tables


% Tightlist

\providecommand{\tightlist}{%
  \setlength{\itemsep}{0pt}\setlength{\parskip}{0pt}}

% URLS
\usepackage[hidelinks]{hyperref}
\usepackage{breakurl}
\usepackage{float} % here for H placement parameter

% Appendix header style

\fancypagestyle{style2}{
\fancyhf{}
\fancyhead[C]{Appendix 1}
}


% For Changing the margins (Education)

\def\changemargin#1#2{\list{}{\rightmargin#2\leftmargin#1}\item[]}
\let\endchangemargin=\endlist 

%For displaying the table in landscape format
\usepackage[absolute]{textpos}

\fancypagestyle{lscape}{% 
\fancyhf{} % clear all header and footer fields 
\fancyfoot[LE]{%
\begin{textblock}{20}(1,5){\rotatebox{90}{\leftmark}}\end{textblock}
\begin{textblock}{1}(13,10.5){\rotatebox{90}{\thepage}}\end{textblock}}
\fancyfoot[LO] {%
\begin{textblock}{1}(13,10.5){\rotatebox{90}{\thepage}}\end{textblock}
\begin{textblock}{20}(1,13.25){\rotatebox{90}{\rightmark}}\end{textblock}}
\renewcommand{\headrulewidth}{0pt} 
\renewcommand{\footrulewidth}{0pt}}

\setlength{\TPHorizModule}{1cm}
\setlength{\TPVertModule}{1cm}

% Appendix
\fancypagestyle{style2}{
\fancyhf{}
\fancyhead[C]{Appendix 1}
\renewcommand{\headrulewidth}{1pt}
}



\newcommand\secbar {
    \tikz[baseline, trim left=3.2cm] 
    {
        \fill [white] (3cm,0.1ex) rectangle +(0.2cm,1.1ex);
        \draw [gray!95, fill=gray!80] (0cm,0.1ex) rectangle (3cm,1.1ex);        
    }
}
\newcommand\subsecbar {
    \tikz[baseline, trim left=0.15cm] 
    {
        \fill [white] (2cm,0.1ex) rectangle +(0.2cm,1.1ex);
        \fill [blue!40] (0cm,0.1ex) rectangle (2cm,1.1ex);      
    }
}

\newcommand\subsubsecbar {
    \tikz[baseline, trim left=0.15cm] 
    {
        \fill [white] (1cm,0.1ex) rectangle +(0.2cm,1.1ex);
        \fill [blue!40] (0cm,0.1ex) rectangle (2cm,1.1ex);      
    }
}

\titleformat{\section}{\large}{}{0cm}{\secbar}
\titleformat{\subsection}{\large}{}{0cm}{\normalfont\sffamily\Large\bfseries\subsecbar}
\titleformat{\subsubsection}{}{}{0cm}{\normalfont\sffamily\large\bfseries}

% No first line paragraph indent
\usepackage{parskip}
\usepackage{enumitem}


\titlespacing\section{0pt}{12pt plus 4pt minus 2pt}{4pt plus 2pt minus 2pt}
\titlespacing\subsection{0pt}{12pt plus 4pt minus 2pt}{4pt plus 2pt minus 2pt}
\titlespacing\subsubsection{0pt}{12pt plus 4pt minus 2pt}{4pt plus 2pt minus 2pt}

\newcolumntype{Y}{>{\centering\arraybackslash}X}

\begin{document}

\centerline{\huge \textbf{Erik Marsja} | \textcolor{darkgray}{Curriculum
Vitae}}

\vspace{2 mm}

\hrule

\begin{table}[h]
\centering
\begin{tabularx}{\textwidth}{@{}lYl@{}}
\textbf{Home Address}: & & 
\\Tvistevägen 26, SE-907 36 Umeå, Sweden & & 
\\\\

 \faPhone \hspace{1 mm}  +4670-36 33
662  \hspace{1 mm}  &  & \faEnvelopeO \hspace{1 mm} \href{mailto:}{\tt \href{mailto:erik@marsja.se}{\nolinkurl{erik@marsja.se}}} \hspace{1 mm}  \\
 \faGlobe \hspace{1 mm} \href{http://www.marsja.se}{\tt www.marsja.se}   &  & \faGithub \hspace{1 mm} \href{http://github.com/marsja}{\tt marsja} \hspace{1 mm}  \\
 \multicolumn{3}{c}{}
 \\\hline
\end{tabularx}
\end{table}

\hypertarget{positions}{%
\subsection{Positions}\label{positions}}

Jan 2022 -

Nov 2022 \hspace{0.75cm}\textbf{Researcher}\vspace{1mm}

\hrule
\begin{changemargin}{2.3cm}{2.4cm}

People in the transport system, Swedish National Road and Transport Research Institute, Linköping, Sweden. Working 80\% of a full-time position.


\end{changemargin}

Jan 2022 -

Nov 2021 \hspace{0.75cm}\textbf{Postdoctoral Researcher}\vspace{1mm}

\hrule
\begin{changemargin}{2.3cm}{2.4cm}

Disability Research Division, Department of Behavioural Sciences and Learning (IBL), Linköping University.

Analyzing data from a large database (i.e., n200) and preparing manuscripts in the field of cognitive hearing science. Conducting qualitative research and about digitalization, listening effort, and work-life (see Funding and Grants). Working 20\% of a full-time position.

\end{changemargin}

Jan 2019 -

Dec 2021\hspace{0.78cm}\textbf{Postdoctoral Researcher}\vspace{1mm}

\hrule
\begin{changemargin}{2.3cm}{2.4cm}

Disability Research Division, Department of Behavioural Sciences and Learning (IBL), Linköping University.

Analyzing data from a large database (i.e., n200) and preparing manuscripts in the field of cognitive hearing science. Responsible for arranging the monthly \href{https://liu.se/linnecentrum-head/en}{HEAD} seminar series (august 2020 - august 2021). This includes contacting national, and international, researchers, booking flights, hotels, as well as setting up equipment for streaming the seminars online.

\end{changemargin}

Aug 2018 -

Oct
2018\hspace{0.75cm}\textbf{Postdoctoral Research Assistant}\vspace{1mm}

\hrule
\begin{changemargin}{2.3cm}{2.4cm}

Department of Psychology, Umeå University.

\textbf{Project}: How does humans perception of safety differ in a broken communication chain compared to an unbroken communication chain? 

Literature study commissioned by SOS Alarm, an emergency agency (e.g., public-safety answering point), with the results disseminating in a report.

\end{changemargin}

Oct 2012 -

June
2018\hspace{0.75cm}\textbf{PhD Student and Teaching Assistant}\vspace{1mm}

\hrule
\begin{changemargin}{2.3cm}{2.4cm}

Department of Psychology, Umeå University.

Planning of studies, programming of experiments, literature search, data analysis, scientific communication, and many more. See the section "Teaching responsibilities" for an overview of my pedagogical experience.

\end{changemargin}

June
2011\hspace{0.75cm}\textbf{Undergrad. Research Assistant}\vspace{1mm}

\hrule
\begin{changemargin}{2.3cm}{2.4cm}

Department of Psychology, Umeå University.

Recruitment of participants and data collection, commenting on later manuscript that ended up in a published paper (see publication 2014). 

\end{changemargin}

\newpage

\hypertarget{degrees-and-titles}{%
\subsection{Degrees and Titles}\label{degrees-and-titles}}

2017 \hspace{1.5cm} \textbf{Ph.D. in Psychology}\vspace{1mm}

\hrule

\begin{changemargin}{2.3cm}{2.4cm}

Department of Psychology, Umeå University.

\textbf{Thesis title}: Attention capture by sudden and unexpected changes: a multisensory perspective. 

Available from the DiVA Database: \sloppy http://umu.diva-portal.org/smash/record.jsf?pid=diva2%3A1156775

\textbf{Supervisors}: Associate Professor Jessica K. Ljungberg, Professor Gregory Neely, \& Dr. Patrik Hansson
\end{changemargin}

2012 \hspace{1.5cm}\textbf{M.Sc. in Cognitive Science}\vspace{1mm}

\hrule
\begin{changemargin}{2.3cm}{2.4cm}

Department of Psychology, Umeå University.

\textbf{Thesis title}: Attention Capture: The Impact of Change in Spatial Sound Source on Behavior. 
    
\textbf{Supervisor}: Associate Professor Jessica K. Ljungberg
\end{changemargin}

2011 \hspace{1.5cm}\textbf{B.Sc. in Cognitive Science}\vspace{1mm}

\hrule

\begin{changemargin}{2.3cm}{2.4cm}

Department of Psychology, Umeå University.

\textbf{Thesis title}:  Attention Capture: Studying the Distracting Effect of One’s Own Name.

Available from DiVA Database: \sloppy http://urn.kb.se/resolve?urn=urn:nbn:se:umu:diva-46607.
    
\textbf{Supervisor}: Associate Professor Jessica K. Ljungberg
\end{changemargin}

\hypertarget{publications}{%
\subsection{Publications}\label{publications}}

\hypertarget{acceptedin-presspublished}{%
\subsubsection{Accepted/In
press/Published}\label{acceptedin-presspublished}}

Stenbäck, V., \textbf{Marsja}, E., Ellis, R., \& Rönnberg, J. (2022).
Relationships between objective and subjective outcome measures of
speech recognition in noise. \emph{International Journal of Audiology}.
\url{https://doi.org/10.1080/14992027.2022.2047232}
\href{https://bit.ly/IJA2022}{(PDF)}

\textbf{Marsja}, E., Stenbäck, V., Moradi, S., Danielsson, H., \&
Rönnberg, J. (2022). Is Having Hearing Loss Fundamentally different?
Multi-group structural equation modeling of the effect of cognitive
functioning on speech identification. \emph{Ear and Hearing}.
\url{https://doi.org/10.1097/AUD.0000000000001196}
\href{https://bit.ly/EANDH22}{(PDF)}

Stenbäck, V., \textbf{Marsja}, E., Hällgren, M., Lyxell, B., \& Larsby,
B. (2021). The Contribution of Age, Working Memory Capacity, and
Inhibitory Control on Speech Recognition in Noise in Young and Older
Adult Listeners. \emph{Journal of Speech, Language, and Hearing
Research, 64}(11), 4513--4523.
\url{https://doi.org/10.1044/2021_JSLHR-20-00251}

Rosa, E., \textbf{Marsja}, E., \& Ljungberg, J. K. (2020). Exploring
Residual Capacity: The Effectiveness of a Vibrotactile Warning During
Increasing Levels of Mental Workload in Simulated Flight Tasks.
\emph{Aviation Psychology and Applied Human Factors, 10}(1), 13--23.
\url{https://doi.org//10.1027/2192-0923/a000180}

\textbf{Marsja}, E., Marsh, J.E., Hansson, P., \& Neely, G. (2019).
Examining the Role of Spatial Changes in Bimodal and Uni-Modal
To-Be-Ignored Stimuli and How They Affect Short-Term Memory Processes.
\emph{Frontiers In Psychology}.
\url{https://doi.org/10.3389/fpsyg.2019.00299}
\href{https://bit.ly/3LkKD19}{(PDF)}

\textbf{Marsja}, E., Neely, G., \& Ljungberg, K.J. (2018). Investigating
Deviance Distraction and the Impact of the Modality of the To-Be-Ignored
Stimuli. \emph{Experimental Psychology, 65}(2), 61--70.
\url{http://doi.org/10.1027/1618-3169/a000390}
\href{https://osf.io/amd2h/.}{(R-Script \& data available:
https://osf.io/amd2h/.)}

Ljungberg, K. J., Parmentier, F. B. R., Jones, D. M., \textbf{Marsja},
E., \& Neely, G. (2014). ``What's in a name?'' ``No more than when it's
mine own''. Evidence from auditory oddball distraction. \emph{Acta
Psychologica, 150C}, 161--166.
\href{http://doi.org/10.1027/1618-3169/a000390}{http://dx.doi.org/10.1016/j.actpsy.2014.05.009}

\hypertarget{in-preparationsubmittedunder-revision}{%
\subsubsection{In Preparation/Submitted/Under
Revision}\label{in-preparationsubmittedunder-revision}}

Stenbäck, V., \textbf{Marsja}, E., Hällgren, M., Lyxell, B., \& Larsby,
B. (Submitted). Informational masking and listening effort in
speech-recognition-in-noise -- the role of working memory capacity and
inhibitory control in older adults with and without hearing impairment.

\textbf{Marsja}, E., Danielsson, H., \& Stenbäck, V. (Manuscript in
Preparation). The Contribution of Cognition to Speech in Noise:
Informational vs.~Energetic Maskers.

\textbf{Marsja}, E. Elbe, P., \& Sörman, D.E. (Manuscript in
Preparation). Examining the Factor Structure of the Swedish Translation
of the Attentional Control Scale and its relation to objective measures
of attention.

Marsh, J.E., Vachon, F., Sörqvist, P., \textbf{Marsja}, E., Röer J.P.,
\& Ljungberg, K.J. (Manuscript in Preparation). Irrelevant vibro-tactile
stimuli produce a changing-state effect: Implications for theories of
interference in short-term memory.

\textbf{Marsja}, E., Neely, G., \& Ljungberg, K.J (Manuscript in
Preparation). Deviance distraction in the auditory and tactile
modalities after repeated exposure: differential aspects of tactile and
auditory deviants.

\hypertarget{conference-presentations}{%
\subsection{Conference Presentations}\label{conference-presentations}}

\textbf{Marsja}, E., Danielsson, H., Stenbäck, V., Moradi, S., Rönnberg,
J. (2019, November). Examining how Cognitive Functioning, Aging, and
Hearing Loss, Affect Speech-in-Noise Performance. Aging and Speech
Communication conference, Tampa, Florida, USA. \textbf{Poster}.

Stenbäck, V., \textbf{Marsja}, E., Danielsson, H., Rönnberg, J. (2019,
November). Logical and Auditory Inference Making: Performance in the
HINT in normally-hearing and hearing-impaired listeners. Aging and
Speech Communication conference, Tampa, Florida, USA. \textbf{Poster}.

Bampouni, E., \textbf{Marsja}, E., Sörman, D.E., \& Ljungberg, K.J
(2019, November). Do Action Gamers Have Enhanced Visual Search Skills? a
Realistic Task Approach. The 15th SweCog conference of the Swedish
Cognitive Science Society, Umeå, Sweden. \textbf{Poster}.

\textbf{Marsja}, E., Danielsson, H., Stenbäck, V., Moradi, S., \&
Rönnberg, J. (2019, June). Examining Relationship Amongst Cognition,
Hearing Loss, Age, \& Speech in Noise. Cognitive Hearing Science for
Communication, Linköping, Sweden. \textbf{Poster}.

Stenbäck, V., Moradi, S., \textbf{Marsja}, E., Danielsson, H., \&
Rönnberg, J. (2019, June). Logical and auditory inference-making in
normally-hearing and hearing-impaired listeners. Cognitive Hearing
Science for Communication, Linköping, Sweden. \textbf{Poster}.

\textbf{Marsja}, E., Marsh, J.E., Neely G., Hansson P., \& Ljungberg
K.J. (2017, April). Domain-generality or domain-specificity of the
short-term memory: insights from a multisensory distraction paradigm.
Re-thinking the Senses Spring School, Dubrovnik, Croatia.
\textbf{Poster}.

\textbf{Marsja}, E., Marsh, J.E., Neely G., Hansson P., \& Ljungberg
K.J. (2016, September). Do Spatial Changes in Sounds and Vibrations
Affect Visuo-spatial and Verbal Short-Term Memory? Attention and
Control: Insights from Distraction, Workshop, University of Central
Lancashire, Preston, UK. \textbf{Invited presentation}.

\textbf{Marsja}, E., Marsh, J.E., Neely G., Hansson P., \& Ljungberg
K.J. (2016, June). Spatial Change in Multisensory Distractors Impact on
Verbal and Spatial Short Term Memory. International Multisensory
Research Forum 17\textsuperscript{th} annual meeting, Suzhou, CHN.
\textbf{Oral presentation}.

\textbf{Marsja}, E., Neely G., Ma, L., \& Ljungberg K.J., (2015,
August). Cross-modality matches of intensity and attention capture
dimensions of auditory and vibrotactile stimuli. Fechner Day 2015. The
31\textsuperscript{st} Annual Meeting of the International Society for
Psychophysics, Québec, CA. \textbf{Poster}.

\textbf{Marsja}, E., Neely, G., Parmentier, F.B.R., \& Ljungberg, K.J.
(2014, October) Deviance Distraction Is Contingent on Stimuli Being
Presented within the Same Modality. Psychonomic Society's
55\textsuperscript{th} Annual Meeting. Long Beach, CA, USA.
\textbf{Poster}.

Ljungberg, K.J., Parmentier, F.B.R., \textbf{Marsja}, E., Neely, \& G.
Jones, D., (2014, January). Any Tom, Dick, or Harry will do: Hearing
one's own name distracts no more than any other in a cross-modal oddball
task. Experimental Psychology Society Meeting. London, UK.
\textbf{Poster}.

\textbf{Marsja}, E., Neely, G., Parmentier, F.B.R., \& Ljungberg, K.J.
(2013, October). Maintenance of the distractive effect of deviating
vibrotactile stimuli in a cross-modal oddball paradigm. The
29\textsuperscript{th} Annual meeting of the International Society of
Psychophysics, Freiburg, DE. \textbf{Poster}.

\hypertarget{popular-science-and-education-related-articles}{%
\subsection{Popular Science and Education-Related
Articles}\label{popular-science-and-education-related-articles}}

\textbf{Marsja}, E. (2016, July). Python Programming in Psychology --
From Data Collection to Analysis. \emph{JEPS Bulletin - The Official
Blog of the Journal of European Psychology Students}. \textbf{Invited
Blog Post}. Retrieved from \sloppy
\url{http://blog.efpsa.org/2016/07/12/python-programming-in-psychology-from-data-collection-to-analysis/}

\hypertarget{science-outreach}{%
\subsection{Science Outreach}\label{science-outreach}}

\textbf{Marsja}, E., \& Stenbäck, V., (June, 2022). Current and Ongoing
Projects on Hearing loss, Cognition, Psychosocial Perspectives, Video
Meetings and Listening Effort. Presentation for the Hearing Clinics
(Öron- näs- och halsklinikerna), Region Östergötland.

\hypertarget{funding-and-grants}{%
\subsection{Funding and Grants}\label{funding-and-grants}}

\textbf{299 235 SEK} from Hörselforskningsfonden (the Hearing Research
Fund) for the project \emph{Lyssningsansträngning hos individer med
hörselnedsättning på grund av ökad digitalisering i arbetslivet
(Listening effort in individuals with hearing impairment due to
increased digitalisation in working life)}, 2021.

\textbf{6000 SEK} from the Department of Psychology, Umeå University,
for participating in the Re-thinking the Senses Spring School,
Dubrovnik, Croatia, 2017.

\textbf{15000 SEK} from Lars Hiertas Minnesfond for the project \emph{Är
korttidsminnet domän-generellt eller domän-specifikt? (Is short-term
memory domain-general or domain specific?)}, 2016.

\textbf{12 000 SEK} from the Faculty of Social sciences, Umeå
University, for participating in the workshop Attention and Control:
Insights from Distraction, and visiting a researcher at the University
of Central Lancashire, Preston, UK, 2016.

\textbf{8000 SEK} from the Department of Psychology, Umeå University,
for participating in the conference 17\textsuperscript{th} International
Multisensory Research Forum 15-18 June, Souzou, China, 2016.

\textbf{10 000 SEK} from JC Kempes minnesfond for the project \emph{Is
everyday distractibility related to attention capture by vibrating
deviants?}, 2014.

\textbf{9 000 SEK} from Knut och Alice Wallenbergs Stiftelse for
participating in the conferences Psychonomic Society's
55\textsuperscript{th} Annual Meeting, 20-23 November, and APCAM, 20
November, Long Beach, USA, 2014.

\textbf{6000 SEK} from the Department of Psychology for participating in
the conference Fechner Day 2013 (the 29\textsuperscript{th} Annual
Meeting of the International Society for Psychophysics) 21-25 October,
Freiburg i.Br., Germany, 2013.

\hypertarget{peer-reviewed-for-journals}{%
\subsection{Peer-reviewed for
Journals:}\label{peer-reviewed-for-journals}}

\begin{itemize}
\item
  Behavior Research Methods
\item
  Frontiers in Psychology
\item
  Journal of Cognitive Psychology
\end{itemize}

Information about my responsibilities as a peer-reviewer can be found in
my \href{https://www.publons.com/a/1517052/}{Publons profile}.

\hypertarget{skills-training}{%
\subsection{Skills \& Training}\label{skills-training}}

\hypertarget{training}{%
\subsubsection{Training}\label{training}}

2017 Spring School \emph{Re-Thinking the Senses}, Inter-University
Centre, Dubrovnik, Croatia

\hypertarget{teaching}{%
\subsubsection{Teaching}\label{teaching}}

Primarily in:

\begin{itemize}
\tightlist
\item
  Research Methods \& Basic Statistics
\item
  Cognitive Psychology (attention, perception, \& working memory)
\item
  Applied Cognitive Psychology (attention and perception)
\end{itemize}

All lectures, seminars, lab demonstrations, supervision of group project
(both involving empirical and applied projects) and supervision of
thesis' have been given at the Department of Psychology, Umeå
University, Sweden (2014 - 2017) and the Department of Behavioral
Science and Learning, Linköping University (2019 - ). At Linköping
University I have also held workshops, and supervised projects. I have
so far taught 950 and 824 clock hours. See Table 1 for an overview of my
teaching responsibilities, clock hours, and courses.

\hypertarget{supervision-of-bachelor-students-undergraduate-level}{%
\subsubsection{Supervision of Bachelor Students (Undergraduate
Level)}\label{supervision-of-bachelor-students-undergraduate-level}}

Jerdhaf, O. (2021). Department of Computer and Information Science,
Linköping University. ``Discovering Implant Terms in Medical Records''.
Bachelor's Thesis in Cognitive Science (18 ECTS). \emph{Main supervisor}

Bridal, O. (2021). Department of Computer and Information Science,
Linköping University. ``Named-entity recognition with BERT for
anonymization of medical records''. Bachelor's Thesis in Cognitive
Science (18 ECTS). \emph{Main supervisor}

Mattila, M. (2021). Department of Computer and Information Science,
Linköping University. ``Synthetic Image Generation Using GANs --
Generating Class-Specific Images of Bacterial Growth''. Bachelor's
Thesis in Cognitive Science (18 ECTS). \emph{Main supervisor}

Rombo, A. (2020). Department of Computer and Information Science,
Linköping University. ``Self-determination perceived by users in support
services pursuant to LSS - An analysis on a municipal level. Bachelor's
Thesis in Cognitive Science (18 ECTS). \emph{Main supervisor}

Lindberg, F. (2020). Department of Computer and Information Science,
Linköping University. ``Hur ungas attityder kring hörselnedsättningar
orsakade av fritidsbuller påverkas av deras koppling till sitt framtida
jag.'' (How young people's attitudes to hearing impairments caused by
leisure noise are affected by their connection to their future selves.)
Bachelor's Thesis in Cognitive Science (18 ECTS). \emph{Main supervisor}

Dahlgren, S. (2020) Department of Computer and Information Science,
Linköping University. ``The association between cognition and
speech-in-noise perception - Investigating the link between
speech-in-noise perception and fluid intelligence in people with and
without hearing loss.'' Bachelor's Thesis in Cognitive Science (18
ECTS). \emph{Main supervisor}

\hypertarget{supervision-of-master-students-graduate-level}{%
\subsubsection{Supervision of Master Students (Graduate
Level)}\label{supervision-of-master-students-graduate-level}}

Carlbring, J. (2020) Department of Computer and Information Science,
Linköping University. ``Inclusive Design for Mobile Devices with WCAG
and Attentional Resources in Mind.'' Master's Thesis in Cognitive
Science (30 ECTS). \emph{Main supervisor}

Ma, L. (2015). Department of Psychology, Umeå University. ``Cross-Modal
Matching of Distractibility in Auditory and Tactile Stimuli''. Master's
Thesis in Cognitive Science (15 ECTS). \emph{Co-supervisor}

Blide, M. (2014). Department of Psychology, Umeå University. ``Att orka
lämna ett misshandelsförhållande: Anknytningens beydelse (To cope
leaving abusive relationships: The importance of attachment). Master's
Thesis in Clinical Psychology (30 ECTS). \emph{Co-supervisor}

\hypertarget{teaching-workshops}{%
\subsubsection{Teaching Workshops}\label{teaching-workshops}}

Oct 2018\hspace{0.75cm}\textbf{R Workshop "Step-by-Step"}\vspace{1mm}

\hrule
\begin{changemargin}{2.15cm}{2.4cm}


Department of Psychology, Umeå University.

An introduction to R statistical programming language - presented to senior researchers with a focus on basic programming and the R environment.

\end{changemargin}

\hypertarget{additional-skills}{%
\subsubsection{Additional skills}\label{additional-skills}}

\begin{itemize}
\tightlist
\item
  Extensive knowledge in statistical software such as SPSS, JASP, and R
  statistical programming environment
\item
  Strong scripting skills in Python (v2.7.x \& v3.x.x) and R
\item
  Substantial skills in programming and performing experiments using
  both E-prime and Python (i.e., PsychoPy, OpenSesame, \& Expyriment)
\item
  Good skills in Microsoft Word and Excel
\item
  Basic programming skills in Visual Basic, E-basic (E-prime), MATLAB,
  Bash, JavaScript, and PHP
\item
  Basic skills in Markdown (e.g., RMarkdown) and \LaTeX
\end{itemize}

\hypertarget{responsibilities}{%
\subsubsection{Responsibilities}\label{responsibilities}}

\begin{itemize}
\tightlist
\item
  Organizer of the monthly HEAD/DRD Seminar Series (e.g., contacting and
  inviting national and international researchers, booking tickets and
  accommodation)
\item
  Elected chairperson of the Ph.D.~students Council of the Department of
  Psychology, Umeå University.
\end{itemize}

\newpage
\pagestyle{empty}

\begin{verbatim}
## Warning in !is.null(rmarkdown::metadata$output) && rmarkdown::metadata$output
## %in% : 'length(x) = 3 > 1' in coercion to 'logical(1)'
\end{verbatim}

\begin{landscape}
\begin{ThreePartTable}
\begin{TableNotes}
\item \textit{Note: } 
\item  Total of clock hours: 1774 , UG = Undergraduate, G = Graduate, VT = Spring term, HT = autumn term, * = two courses.
\end{TableNotes}
\begin{longtable}[t]{l>{\raggedright\arraybackslash}p{5cm}>{\raggedright\arraybackslash}p{5cm}l>{\raggedright\arraybackslash}p{5cm}ll}
\caption{\label{tab:unnamed-chunk-5}Teaching responsibilities - an overview of type of teaching, hours, etc.}\\
\toprule
Period & Subject & Type & Clock Hours & Course/Program & Level & Language\\
\midrule
\endfirsthead
\caption[]{Teaching responsibilities - an overview of type of teaching, hours, etc. \textit{(continued)}}\\
\toprule
Period & Subject & Type & Clock Hours & Course/Program & Level & Language\\
\midrule
\endhead

\endfoot
\bottomrule
\insertTableNotes
\endlastfoot
VT 2014 & Scientific communication, research methods, research ethics & Lectures, seminars, supervision & 60 & Psychological test and research methods/Statistical and Empirial Methods, the Clinical Psychologist program/the Bachelor's Program in Cognitive Science & UG & Swe\\
 & Cognitive Psychology & Supervision of projects & 20 & Introduction to Psychology, the Clincial Psychologist Program & UG & \vphantom{1} Swe\\
 &  &  &  &  &  \vphantom{13} & \\
HT 2014 & Clinical Psychology & Co-supervision of thesis project & 20 & Master Thesis in Psychology, 30 ECTS, the Clinical Psychologist Program & G & Eng\\
 & Applied Cognitive Science & Supervision of projects & 40 & Project in Cognitive Science, the Bachelor's Program in Cognitive Science & UG & \vphantom{1} Swe\\
\addlinespace
 & Cognitive Psychology & Supervision of projects & 20 & Introduction to Psychology, the Clincial Psychologist Program & UG & Swe\\
 &  &  &  &  &  \vphantom{12} & \\
VT 2015 & Cognitive Psychology & Supervision of projects & 24 & Introduction to Psychology, the Clincial Psychologist Program & UG & Swe\\
 & Cognitive Psychology & Lectures, seminars, supervision of projects & 20 & Basic psychology and sport psychology, the Coaching Program & UG & Swe\\
 & Cognitive Psychology & Supervision of thesis project & 15 & Master Thesis in Cognitive Science, 15 ECTS, the Master's Program in Cognitive Science & G & Eng\\
\addlinespace
 & Applied Cognitive Science & Supervision of projects & 110 & Project in Cognitive Science, the Bachelor's Program in Cognitive Science & UG & \vphantom{1} Swe\\
 &  &  &  &  &  \vphantom{11} & \\
HT 2015 & Cognitive Psychology & Supervision of projects & 24 & Cognitive Psychology, the Clinical Psychologist Program & UG & Swe\\
 & Qualitative methods & Seminars & 20 & the Psychology of Organizations, Environment and Work, the Clinical Psychologist Program & UG & Swe\\
 &  &  &  &  &  \vphantom{10} & \\
\addlinespace
VT 2016 & Cognitive Psychology & Lab demonstrations & 30 & 2:1 Cognition, the Clinical Psychologist Program & UG & Eng\\
 & Cognitive Psychology: Attention & Lectures, seminars, supervision of projects & 60 & Applied Cognitive Psychology, the Bachelor's Program in Cognitive Science & UG & Swe\\
 & Applied Cognitive Science & Supervision of projects & 110 & Project in Cognitive Science, the Bachelor's Program in Cognitive Science & UG & Swe\\
 & Qualitative and Quantitative methods & Seminars, computer labs & 30 & the Psychology of Organizations, Environment and Work, the Clinical Psychologist Program & UG & \vphantom{1} Swe\\
 &  &  &  &  &  \vphantom{9} & \\
\addlinespace
HT 2016 & Cognitive Psychology & Lab demonstrations & 30 & 2:1 Cognition, the Clinical Psychologist Program & UG & Eng\\
 & Qualitative and Quantitative methods & Seminars, computer labs & 30 & the Psychology of Organizations, Environment and Work, the Clinical Psychologist Program & UG & Swe\\
 & Cognitive Psychology: Perception, Attention, and Consciousness & Lectures & 24 & Perception, the Bachelor's Program in Cognitive Science & UG & Swe\\
 & Clinical Psychology & Seminars & 9 & Master Thesis in Psychology, 30 ECTS, the Clinical Psychologist Program & G & Swe\\
 &  &  &  &  &  \vphantom{8} & \\
\addlinespace
HT 2017 & Cognitive Psychology & Lab demonstrations & 30 & 2:1 Cognition, the Clinical Psychologist Program & UG & Eng\\
 &  &  &  &  &  \vphantom{7} & \\
VT 2017 & Cognitive Psychology, Applied: Attention & Lectures, seminar & 20 & Applied Cognitive Psychology, the Bachelor's Program in Cognitive Science & UG & Swe\\
 & Applied Cognitive Science & Supervision of projects & 40 & Project in Cognitive Science, the Bachelor's Program in Cognitive Science & UG & Swe\\
 & Qualatitive and Quantative research methods & Seminars, computer labs & 30 & the Psychology of Organizations, Environment and Work, the Clinical Psychologist Program & UG & Swe\\
\addlinespace
 & Clinical Psychology & Seminars & 30 & Master Thesis in Psychology, 30 ECTS, the Clinical Psychologist Program & G & Swe\\
 & Cognitive Psychology & Lab demonstrations & 30 & 2:1 Cognition, the Clinical Psychologist Program & UG & Eng\\
 & Cognitive Psychology: Attention and Consciousness & Lectures, seminars & 24 & Perception, the Bachelor's Program in Cognitive Science & UG & Swe\\
 & Developmental Psychology & Seminar leader & 50 & Learning and teaching, pedagogy & UG & Swe\\
 &  &  &  &  &  \vphantom{6} & \\
\addlinespace
VT 2019 & Research Methods & Workshops in SPSS & 18 & Human Resources, Special Pedagogists & G & Swe\\
 & Applied Cognitive Science & Supervision of Projects & 24 & Applied Cognitive Science, the Bachelor's Program in Cognitive Science & UG & Eng\\
 & Pedagogy & Examination of Thesis & 6 & Masther Thesis in Special Pedagogy, 30 ECTS, the Specialist Pedagogy Program & G & Swe\\
 &  &  &  &  &  \vphantom{5} & \\
HT 2019 & Quantative methods, theory of science & Seminar leader, grading exams & 62 & Theory of Science and Research Methods, Pedagogy & UG & Swe\\
\addlinespace
 & Cognitive Science & Seminar leader, grading exams & 20 & Cognitive Science - Methods, the Master's Program in Cognitive Science & G & Swe\\
 & Disability Research & Lectures, workshops, examining & 40 & Theory of Science and Research Methods, Disability Research II and III & UG & Swe\\
 & Perception & Lectures & 12 & Neuropsychology and Neuroscience, the Clinical Psychologist Program & G & Swe\\
 & Applied Learning Disabilities & Supervision of groups (Problem-based learning) & 16 & Cognitive Psychology, the Clinical Psychologists Program & UG & \vphantom{1} Swe\\
 & Adult life and Aging & Supervision of groups (Problem-based learning) & 20 & Developmental Psychology, the Clinical Psychologists Program & UG & \vphantom{1} Swe\\
\addlinespace
 &  &  &  &  &  \vphantom{4} & \\
HT and VT 2020 & Quantative methods & Lectures & 92 & Methods for Research and Development in Educational Institutions, Pedagogy & G & Swe\\
 &  &  &  &  &  \vphantom{3} & \\
VT 2020 & Applied Cognitive Science & Supervision & 24 & Applied Cognitive Science, the Bachelor's Program in Cognitive Science & UG & Eng\\
 & Cognitive Science & Supervision, Examinating & 45 & Bachelor's Thesis in Cognitive Science, the Bachelor's Program in Cognitive Science & UG & \vphantom{1} Swe\\
\addlinespace
 & Cognitive Science & Supervision & 20 & Masther's Thesis in Cognitive Science, the Master's Program in Cognitive Science & G & Swe\\
 &  &  &  &  &  \vphantom{2} & \\
HT 2020 & Disability Research & Lectures, workshops, examining & 80 & Theory of Science and Research Methods, Disability Research II and III & UG & Swe\\
 & Quantative methods, theory of science & Seminar leader, grading exams & 50 & Theory of Science and Research Methods, Pedagogy & UG & Swe\\
 &  &  &  &  &  \vphantom{1} & \\
\addlinespace
VT 2021 & Applied Cognitive Science & Supervision & 20 & Applied Cognitive Science, the Bachelor's Program in Cognitive Science & UG & Eng\\
 &  & Supervision of research project & 8 & Research Project, the Master's Programme in Cognitive Science & G & Swe\\
 & Cognitive Science & Supervision, Examinating & 45 & Bachelor's Thesis in Cognitive Science, the Bachelor's Program in Cognitive Science & UG & Swe\\
 & Disability Research & Supervision of theses & 30 & University Diploma Thesis, Disability Research II & UG & Swe\\
 &  &  &  &  &  & \\
\addlinespace
HT 2021 & Disability Research & Lectures, workshops, examining & 80 & Theory of Science and Research Methods, Disability Research II and III & UG & Swe\\
 & Disability Research & Lectures, examination & 40 & Introduction to Research methods and Ethics & UG & Swe\\
 & Research Methods & Supervision of groups (Problem-based learning) & 36 & Research Methods, the Clinical Psychologists Program & UG & Swe\\
 & Applied Learning Disabilities & Supervision of groups (Problem-based learning) & 16 & Cognitive Psychology, the Clinical Psychologists Program & UG & Swe\\
 & Adult life and Aging & Supervision of groups (Problem-based learning) & 20 & Developmental Psychology, the Clinical Psychologists Program & UG & Swe\\*
\end{longtable}
\end{ThreePartTable}
\end{landscape}

\end{document}