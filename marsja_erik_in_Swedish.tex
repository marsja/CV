\documentclass[]{article}
\usepackage{pdflscape}

\usepackage{everypage}

\newcommand{\Lpagenumber}{\ifdim\textwidth=\linewidth\else\bgroup
  \dimendef\margin=0 %use \margin instead of \dimen0
  \ifodd\value{page}\margin=\oddsidemargin
  \else\margin=\evensidemargin
  \fi
  \raisebox{\dimexpr -\topmargin-\headheight-\headsep-0.5\linewidth}[0pt][0pt]{%
    \rlap{\hspace{\dimexpr \margin+\textheight+\footskip}%
    \llap{\rotatebox{90}{ CV  - Erik Marsja 
- \thepage\hspace{1pt}(\pageref{LastPage})}}}}%
\egroup\fi}
\AddEverypageHook{\Lpagenumber}%

\usepackage{adjustbox}
\usepackage{titling}
\usepackage{caption}
\usepackage{tabularx}
\usepackage{tikz}
\usepackage{xcolor}
\usepackage{fancyhdr}
\usepackage{lastpage}
\usepackage{titlesec}
\usepackage[scaled=.90]{helvet}% Helvetica, served as a model for arial

% Added for new tabl:
\usepackage{booktabs}
\usepackage{threeparttablex}
\usepackage{longtable}


\pagestyle{fancy}
\fancyhf{}
\renewcommand{\headrulewidth}{0pt}
\renewcommand{\footrulewidth}{0.0pt}
\fancyfoot[CO,CE]{  CV -   Erik Marsja -  \thepage\hspace{1pt}(\pageref{LastPage})}
\fancypagestyle{plain}{\pagestyle{fancy}}

\usepackage[margin=1.2in]{geometry}

% Font awesome
\usepackage{fontawesome}

\usepackage[T1]{fontenc}
\usepackage[utf8]{inputenc}

% For tables


% Tightlist

\providecommand{\tightlist}{%
  \setlength{\itemsep}{0pt}\setlength{\parskip}{0pt}}

% URLS
\usepackage[hidelinks]{hyperref}
\usepackage{breakurl}
\usepackage{float} % here for H placement parameter

% Appendix header style

\fancypagestyle{style2}{
\fancyhf{}
\fancyhead[C]{Appendix 1}
}


% For Changing the margins (Education)

\def\changemargin#1#2{\list{}{\rightmargin#2\leftmargin#1}\item[]}
\let\endchangemargin=\endlist 

%For displaying the table in landscape format
\usepackage[absolute]{textpos}

\fancypagestyle{lscape}{% 
\fancyhf{} % clear all header and footer fields 
\fancyfoot[LE]{%
\begin{textblock}{20}(1,5){\rotatebox{90}{\leftmark}}\end{textblock}
\begin{textblock}{1}(13,10.5){\rotatebox{90}{\thepage}}\end{textblock}}
\fancyfoot[LO] {%
\begin{textblock}{1}(13,10.5){\rotatebox{90}{\thepage}}\end{textblock}
\begin{textblock}{20}(1,13.25){\rotatebox{90}{\rightmark}}\end{textblock}}
\renewcommand{\headrulewidth}{0pt} 
\renewcommand{\footrulewidth}{0pt}}

\setlength{\TPHorizModule}{1cm}
\setlength{\TPVertModule}{1cm}

% Appendix
\fancypagestyle{style2}{
\fancyhf{}
\fancyhead[C]{Appendix 1}
\renewcommand{\headrulewidth}{1pt}
}



\newcommand\secbar {
    \tikz[baseline, trim left=3.2cm] 
    {
        \fill [white] (3cm,0.1ex) rectangle +(0.2cm,1.1ex);
        \draw [gray!95, fill=gray!80] (0cm,0.1ex) rectangle (3cm,1.1ex);        
    }
}
\newcommand\subsecbar {
    \tikz[baseline, trim left=0.15cm] 
    {
        \fill [white] (2cm,0.1ex) rectangle +(0.2cm,1.1ex);
        \fill [blue!40] (0cm,0.1ex) rectangle (2cm,1.1ex);      
    }
}

\newcommand\subsubsecbar {
    \tikz[baseline, trim left=0.15cm] 
    {
        \fill [white] (1cm,0.1ex) rectangle +(0.2cm,1.1ex);
        \fill [blue!40] (0cm,0.1ex) rectangle (2cm,1.1ex);      
    }
}

\titleformat{\section}{\large}{}{0cm}{\secbar}
\titleformat{\subsection}{\large}{}{0cm}{\normalfont\sffamily\Large\bfseries\subsecbar}
\titleformat{\subsubsection}{}{}{0cm}{\normalfont\sffamily\large\bfseries}

% No first line paragraph indent
\usepackage{parskip}
\usepackage{enumitem}


\titlespacing\section{0pt}{12pt plus 4pt minus 2pt}{4pt plus 2pt minus 2pt}
\titlespacing\subsection{0pt}{12pt plus 4pt minus 2pt}{4pt plus 2pt minus 2pt}
\titlespacing\subsubsection{0pt}{12pt plus 4pt minus 2pt}{4pt plus 2pt minus 2pt}

\newcolumntype{Y}{>{\centering\arraybackslash}X}

\begin{document}

\centerline{\huge \textbf{Erik Marsja} | \textcolor{darkgray}{Curriculum Vitae}}

\vspace{2 mm}

\hrule

\begin{table}[h]
\centering
\begin{tabularx}{\textwidth}{@{}lYl@{}}
\textbf{Address}: & & 
\\Tvistevägen 26, SE-907 36 Umeå, Sweden & & 
\\\\

 \faPhone \hspace{1 mm}  070-36 33 662  \hspace{1 mm}  &  & \faEnvelopeO \hspace{1 mm} \href{mailto:}{\tt \href{mailto:erik.marsja@liu.se}{\nolinkurl{erik.marsja@liu.se}}} \hspace{1 mm}  \\
 \faGlobe \hspace{1 mm} \href{http://www.marsja.se}{\tt www.marsja.se}   &  & \faGithub \hspace{1 mm} \href{http://github.com/marsja}{\tt marsja} \hspace{1 mm}  \\
 \multicolumn{3}{c}{}
 \\\hline
\end{tabularx}
\end{table}

\hypertarget{anstuxe4llningar}{%
\subsection{Anställningar}\label{anstuxe4llningar}}

Jan 2022 - \hspace{0.50cm}\textbf{Forskare}\vspace{1mm}

\hrule
\begin{changemargin}{2.3cm}{2.4cm}

Väg- och transportinstitutet (VTI), Linköpings Universitet.

\end{changemargin}

Jan 2019 -

dec 2021\hspace{0.75cm}\textbf{Postdoktoral Forskare}\vspace{1mm}

\hrule
\begin{changemargin}{2.3cm}{2.4cm}

Avdelningen för Handikappvetenskap, Institutionen för Beteendevetenskap och Lärande, Linköpings Universitet.

\textbf{Arbetsuppgifter}: Analysera data från en stor databas och skriva manuskript inom fältet för kognitiv hörselvetenskap. Ansvarig för att arrangera den månatliga seminarieserien \href{https://liu.se/linnecentrum-head/en}{HEAD} seminar series (augusti 2020 - augusti 2021). Detta inkluderade att kontakta nationella, och internationella, forskare, boka flyg och boende, såväl som lokaler för seminariet och att se till så att seminarierna gick att sända online. Undervisning i framför allt statistik och metod (kvantitativ såväl som kvalitativ).

\end{changemargin}

Aug 2018 -

okt 2018\hspace{0.75cm}\textbf{Forskarassistent}\vspace{1mm}

\hrule
\begin{changemargin}{2.3cm}{2.4cm}

Institutionen för Psykologi, Umeå Universitet.

\textbf{Projekttitel}: Upplevd säkerhet och trygghet i nöd- och vårdsituationer.

Gjorde en litteraturstudie på uppdrag av SOS Alarm. Denna litteraturstudie innefattade främst studier av kvalitativ karaktär med inriktning på vilka faktorer som kan påverka känslan av trygghet och säkerhet vid både nöd- och vårdsituationer. Resultatet presenterades i rapport form och muntligt för SOS Alarms innovationsstyrelse.

\end{changemargin}

Okt 2012 -

juni 2018\hspace{0.75cm}\textbf{Doktorand}\vspace{1mm}

\hrule
\begin{changemargin}{2.3cm}{2.4cm}

Institutionen för Psykologi, Umeå Universitet.

Planering av studier, programmering av experiment, litteratursökning, dataanalys, vetenskaplig kommunikation och många fler. Se avsnittet "Undervisningsansvar" för en översikt över min pedagogiska erfarenhet.

\end{changemargin}

Juni 2011\hspace{0.75cm}\textbf{Projektassistent}\vspace{1mm}

\hrule
\begin{changemargin}{2.3cm}{2.4cm}

Institutionen för Psykologi, Umeå Universitet.

Rekryterade deltagare och samlade in data.

\end{changemargin}

\hypertarget{examina}{%
\subsection{Examina}\label{examina}}

2017 \hspace{1.5cm} \textbf{Fil. Dr. i Psykologi}\vspace{1mm}

\hrule

\begin{changemargin}{2.3cm}{2.4cm}

Institutionen för Psykologi, Umeå Universitet.

\textbf{Avhandlingens titel}: Attention capture by sudden and unexpected changes: a multisensory perspective. 

Tillgänglig från DiVA: \sloppy http://umu.diva-portal.org/smash/record.jsf?pid=diva2%3A1156775

\textbf{Handledare}: Professor Jessica K. Ljungberg, Professor Gregory Neely, \& Dr. Patrik Hansson
\end{changemargin}

2012 \hspace{1.5cm}\textbf{M.Sc. i Kognitionsvetenskap}\vspace{1mm}

\hrule
\begin{changemargin}{2.3cm}{2.4cm}

Institutionen för Psykologi, Umeå Universitet.

\textbf{Uppsatstitel}: Attention Capture: The Impact of Change in Spatial Sound Source on Behavior. 
    
\textbf{Handledare}: Professor Jessica K. Ljungberg
\end{changemargin}

2011 \hspace{1.5cm}\textbf{B.Sc. i Kognitionsvetenskap}\vspace{1mm}

\hrule

\begin{changemargin}{2.3cm}{2.4cm}

Institutionen för Psykologi, Umeå Universitet.

\textbf{Uppsatstitel}:  Attention Capture: Studying the Distracting Effect of One’s Own Name.

Tillgänglig från DiVA: \sloppy http://urn.kb.se/resolve?urn=urn:nbn:se:umu:diva-46607.
    
\textbf{Supervisor}: Professor Jessica K. Ljungberg
\end{changemargin}

\hypertarget{publikationslista}{%
\subsection{Publikationslista}\label{publikationslista}}

\hypertarget{internationalla-refereegranskade-tidskrifter}{%
\subsubsection{Internationalla refereegranskade
tidskrifter}\label{internationalla-refereegranskade-tidskrifter}}

\textbf{Marsja}, E., Stenbäck, V., Moradi, S., Danielsson, H., \&
Rönnberg, J. (Accepted). Is Having Hearing Loss Fundamentally different?
Multi-group structural equation modeling of the effect of cognitive
functioning on speech identification. \emph{Ear and Hearing}.

Stenbäck, V., \textbf{Marsja}, E., Hällgren, M., Lyxell, B., \& Larsby,
B. (2021). The Contribution of Age, Working Memory Capacity, and
Inhibitory Control on Speech Recognition in Noise in Young and Older
Adult Listeners. \emph{Journal of Speech, Language, and Hearing
Research, 64}(11), 4513--4523.
\url{https://doi.org/10.1044/2021_JSLHR-20-00251}

Rosa, E., \textbf{Marsja}, E., \& Ljungberg, J. K. (2020). Exploring
Residual Capacity: The Effectiveness of a Vibrotactile Warning During
Increasing Levels of Mental Workload in Simulated Flight Tasks.
\emph{Aviation Psychology and Applied Human Factors, 10(1)}, 13--23.
\url{https://doi.org//10.1027/2192-0923/a000180}

\textbf{Marsja}, E., Marsh, J.E., Hansson, P., \& Neely, G. (2019).
Examining the Role of Spatial Changes in Bimodal and Uni-Modal
To-Be-Ignored Stimuli and How They Affect Short-Term Memory Processes.
\emph{Frontiers In Psychology}.
\url{https://doi.org/10.3389/fpsyg.2019.00299}

\textbf{Marsja}, E., Neely, G., \& Ljungberg, K.J. (2018). Investigating
Deviance Distraction and the Impact of the Modality of the To-Be-Ignored
Stimuli. \emph{Experimental Psychology, 65}(2), 61--70.
\url{http://doi.org/10.1027/1618-3169/a000390}

Ljungberg, K. J., Parmentier, F. B. R., Jones, D. M., \textbf{Marsja},
E., \& Neely, G. (2014). ``What's in a name?'' ``No more than when it's
mine own''. Evidence from auditory oddball distraction. \emph{Acta
Psychologica, 150C}, 161--166.
\url{http://dx.doi.org/10.1016/j.actpsy.2014.05.009}

\hypertarget{publikationer-under-fuxf6rberedelseinskickadeunder-revidering}{%
\subsubsection{Publikationer under Förberedelse/inskickade/under
revidering}\label{publikationer-under-fuxf6rberedelseinskickadeunder-revidering}}

Stenbäck, V., \textbf{Marsja}, E., Ellis, R., \& Rönnberg, J. (Under
Review). Relationships between objective and subjective outcome measures
of speech recognition in noise.

Stenbäck, V., \textbf{Marsja}, E., Hällgren, M., Lyxell, B., \& Larsby,
B. (Manuscript in Preparation). Informational masking and listening
effort in speech-recognition-in-noise -- the role of working memory
capacity and inhibitory control in older adults with and without hearing
impairment.

\textbf{Marsja}, E., Danielsson, H., \& Stenbäck, V. (Manuscript in
Preparation). The Contribution of Cognition to Speech in Noise:
Informational vs.~Energetic Maskers.

\textbf{Marsja}, E. Elbe, P., \& Sörman, D.E. (Manuscript in
Preparation). Examining the Factor Structure of the Swedish Translation
of the Attentional Control Scale and its relation to objective measures
of attention.

Marsh, J.E., Vachon, F., Sörqvist, P., \textbf{Marsja}, E., Röer J.P.,
\& Ljungberg, K.J. (Manuscript in Preparation). Irrelevant vibro-tactile
stimuli produce a changing-state effect: Implications for theories of
interference in short-term memory.

\textbf{Marsja}, E., Neely, G., \& Ljungberg, K.J (Manuscript in
Preparation). Deviance distraction in the auditory and tactile
modalities after repeated exposure: differential aspects of tactile and
auditory deviants.

\hypertarget{konferenspresentationer}{%
\subsection{Konferenspresentationer}\label{konferenspresentationer}}

\textbf{Marsja}, E., Danielsson, H., Stenbäck, V., Moradi, S., Rönnberg,
J. (2019, November). Examining how Cognitive Functioning, Aging, and
Hearing Loss, Affect Speech-in-Noise Performance. Aging and Speech
Communication conference, Tampa, Florida, USA. \textbf{Poster}.

Stenbäck, V., \textbf{Marsja}, E., Danielsson, H., Rönnberg, J. (2019,
November). Logical and Auditory Inference Making: Performance in the
HINT in normally-hearing and hearing-impaired listeners. Aging and
Speech Communication conference, Tampa, Florida, USA. \textbf{Poster}.

Bampouni, E., \textbf{Marsja}, E., Sörman, D.E., \& Ljungberg, K.J
(2019, November). Do Action Gamers Have Enhanced Visual Search Skills? a
Realistic Task Approach. The 15th SweCog conference of the Swedish
Cognitive Science Society, Umeå, Sweden. \textbf{Poster}.

\textbf{Marsja}, E., Danielsson, H., Stenbäck, V., Moradi, S., \&
Rönnberg, J. (2019, June). Examining Relationship Amongst Cognition,
Hearing Loss, Age, \& Speech in Noise. Cognitive Hearing Science for
Communication, Linköping, Sweden. \textbf{Poster}.

Stenbäck, V., Moradi, S., \textbf{Marsja}, E., Danielsson, H., \&
Rönnberg, J. (2019, June). Logical and auditory inference-making in
normally-hearing and hearing-impaired listeners. Cognitive Hearing
Science for Communication, Linköping, Sweden. \textbf{Poster}.

\textbf{Marsja}, E., Marsh, J.E., Neely G., Hansson P., \& Ljungberg
K.J. (2017, April). Domain-generality or domain-specificity of the
short-term memory: insights from a multisensory distraction paradigm.
Re-thinking the Senses Spring School, Dubrovnik, Croatia.
\textbf{Poster}.

\textbf{Marsja}, E., Marsh, J.E., Neely G., Hansson P., \& Ljungberg
K.J. (2016, September). Do Spatial Changes in Sounds and Vibrations
Affect Visuo-spatial and Verbal Short-Term Memory? Attention and
Control: Insights from Distraction, Workshop, University of Central
Lancashire, Preston, UK. \textbf{Inbjuden presentation}.

\textbf{Marsja}, E., Marsh, J.E., Neely G., Hansson P., \& Ljungberg
K.J. (2016, June). Spatial Change in Multisensory Distractors Impact on
Verbal and Spatial Short Term Memory. International Multisensory
Research Forum 17\textsuperscript{th} annual meeting, Suzhou, CHN.
\textbf{Oral presentation}.

\textbf{Marsja}, E., Neely G., Ma, L., \& Ljungberg K.J., (2015,
August). Cross-modality matches of intensity and attention capture
dimensions of auditory and vibrotactile stimuli. Fechner Day 2015. The
31\textsuperscript{st} Annual Meeting of the International Society for
Psychophysics, Québec, CA. \textbf{Poster}.

\textbf{Marsja}, E., Neely, G., Parmentier, F.B.R., \& Ljungberg, K.J.
(2014, October) Deviance Distraction Is Contingent on Stimuli Being
Presented within the Same Modality. Psychonomic Society's
55\textsuperscript{th} Annual Meeting. Long Beach, CA, USA.
\textbf{Poster}.

Ljungberg, K.J., Parmentier, F.B.R., \textbf{Marsja}, E., Neely, \& G.
Jones, D., (2014, January). Any Tom, Dick, or Harry will do: Hearing
one's own name distracts no more than any other in a cross-modal oddball
task. Experimental Psychology Society Meeting. London, UK.
\textbf{Poster}.

\textbf{Marsja}, E., Neely, G., Parmentier, F.B.R., \& Ljungberg, K.J.
(2013, October). Maintenance of the distractive effect of deviating
vibrotactile stimuli in a cross-modal oddball paradigm. The
29\textsuperscript{th} Annual meeting of the International Society of
Psychophysics, Freiburg, DE. \textbf{Poster}.

\hypertarget{populuxe4rvetenskapliga-och-utbildningsrelaterade-artiklar}{%
\subsection{Populärvetenskapliga och Utbildningsrelaterade
Artiklar}\label{populuxe4rvetenskapliga-och-utbildningsrelaterade-artiklar}}

\textbf{Marsja}, E. (2016, juli). Python Programming in Psychology --
From Data Collection to Analysis. \emph{JEPS Bulletin - The Official
Blog of the Journal of European Psychology Students}. \textbf{Invited
Blog Post}. Retrieved from \sloppy
\url{http://blog.efpsa.org/2016/07/12/python-programming-in-psychology-from-data-collection-to-analysis/}

\hypertarget{finansiering-och-bidrag}{%
\subsection{Finansiering och bidrag}\label{finansiering-och-bidrag}}

\textbf{299 235 SEK} från Hörselforskningsfonden för projektet
\emph{Lyssningsansträngning hos individer med hörselnedsättning på grund
av ökad digitalisering i arbetslivet}, 2021.

\textbf{6000 SEK} från Institutionen för Psykologi, Umeå Universitet,
för att delta i ``Re-thinking the Senses Spring School'', Dubrovnik,
Kroatien, 2017.

\textbf{15000 SEK} från Lars Hiertas Minnesfond för projektet \emph{Är
korttidsminnet domän-generellt eller domän-specifikt? (Is short-term
memory domain-general or domain specific?)}, 2016.

\textbf{12 000 SEK} från Samhällsvetenskapliga fakulteten, Umeå
Universitet, för att delta i workshop:Attention and Control: Insights
from Distraction, och för att besöka Dr.~John E. Marsh vid the
University of Central Lancashire, Preston, UK, 2016.

\textbf{8000 SEK} från Institutionen för Psykologi, Umeå Universitet,
för att delta i konferensen the 17\textsuperscript{th} International
Multisensory Research Forum 15-18 juni, Souzou, Kina, 2016.

\textbf{10 000 SEK} från JC Kempes minnesfond för projektet \emph{Is
everyday distractibility related to attention capture by vibrating
deviants?}, 2014.

\textbf{9 000 SEK} från Knut och Alice Wallenbergs Stiftelse för att
delta i konferenserna Psychonomic Society's 55\textsuperscript{th}
Annual Meeting, 20-23 november, och APCAM, 20:e november, Long Beach,
USA, 2014.

\textbf{6000 SEK} från Institutionen för Psykologi, Umeå Universitet,
för att delta i Fechner Day 2013 (the 29\textsuperscript{th} Annual
Meeting of the International Society for Psychophysics) 21-25 October,
Freiburg i.Br., Tyskland, 2013.

\hypertarget{refereegranskat-fuxf6r-tidskrifter}{%
\subsection{Refereegranskat för
tidskrifter:}\label{refereegranskat-fuxf6r-tidskrifter}}

\begin{itemize}
\item
  Behavior Research Methods
\item
  Frontiers in Psychology
\item
  Journal of Cognitive Psychology
\end{itemize}

Information om mitt ansvar som granskare finns i min
{[}Publons-profil{]} (\url{https://www.publons.com/a/1517052/}).

\hypertarget{specialisttruxe4ning}{%
\subsection{Specialistträning}\label{specialisttruxe4ning}}

2017 Spring School \emph{Re-Thinking the Senses}, Inter-University
Centre, Dubrovnik, Croatia

\hypertarget{undervisningsansvar}{%
\subsection{Undervisningsansvar}\label{undervisningsansvar}}

Primärt i:

\begin{itemize}
\tightlist
\item
  Forskningsmetodik, grundläggande statistik och kvalitativ metod
\item
  Kognitiv Psykologi
\item
  Tillämpad Kognitiv Psykologi
\end{itemize}

Majoriteten av föreläsningar, seminarier, laboratoriedemonstrationer,
handledning av gruppprojekt (både som involverar empiriska och
tillämpade projekt) och handledning av uppsatser har skett vid
Institutionen för Psykologi, Umeå universitet, Sverige (2014 - 2017) och
Institutionen för beteendevetenskap och lärande, Linköpings universitet
(IBL: 2019 -). Jag har också hållit workshops vid IBL, Linköpings
universitet (2019 -). Vid Umeå universitet och Linköpings universitet
har jag hittills undervisat i 950 respektive 550 timmar. Se Tabell 1 för
en översikt över mina undervisningsansvar, antal timmar och kurser.

\hypertarget{handledning-av-kandidatstudenter-grundnivuxe5}{%
\subsubsection{Handledning av Kandidatstudenter
(grundnivå)}\label{handledning-av-kandidatstudenter-grundnivuxe5}}

Rombo, A. (2020). Institutionen för dator- och informationsvetenskap,
Linköpings universitet. "Self-determination perceived by users in
support services pursuant to LSS - An analysis on a municipal level.
Kandidatuppsats i Kognitionsvetenskap (18HP). \emph{Huvudhandledare}

Lindberg, F. (2020). Institutionen för dator- och informationsvetenskap,
Linköpings universitet. ``Hur ungas attityder kring hörselnedsättningar
orsakade av fritidsbuller påverkas av deras koppling till sitt framtida
jag.'' Kandidatuppsats i Kognitionsvetenskap (18HP).
\emph{Huvudhandledare}

Dahlgren, S. (2020). Institutionen för dator- och informationsvetenskap,
Linköpings universitet. ``The association between cognition and
speech-in-noise perception - Investigating the link between
speech-in-noise perception and fluid intelligence in people with and
without hearing loss.'' Kandidatuppsats i Kognitionsvetenskap (18HP).
\emph{Huvudhandledare}

\hypertarget{handledning-av-masterstudenter-avancerad-nivuxe5}{%
\subsubsection{Handledning av Masterstudenter (avancerad
nivå)}\label{handledning-av-masterstudenter-avancerad-nivuxe5}}

Carlbring, J. (2020) Institutionen för dator- och informationsvetenskap,
Linköpings universitet.. ``Inclusive Design for Mobile Devices with WCAG
and Attentional Resources in Mind.'' Masteruppsats i Kognitionsvetenskap
(30hp). \emph{Huvudhandledare}

Ma, L. (2015). Institutionen för Psykologi, Umeå Universitet.
``Cross-Modal Matching of Distractibility in Auditory and Tactile
Stimuli''. Masteruppsats i Kognitionsvetenskap (15HP).
\emph{Medhandledare}

Blide, M. (2014). Institutionen för Psykologi, Umeå Universitet. "Att
orka lämna ett misshandelsförhållande: Anknytningens beydelse.
Masteruppsats i Klinisk Psykologi (30HP). \emph{Medhandledare}

\hypertarget{workshops}{%
\subsubsection{Workshops}\label{workshops}}

Okt 2018\hspace{0.75cm}\textbf{R Workshop "Step-by-Step"}\vspace{1mm}

\hrule
\begin{changemargin}{2.15cm}{2.4cm}


Institutionen för Psykologi, Umeå Universitet.

En introduktion till R statistiskt programmeringsspråk - presenterad för seniorforskare med fokus på grundläggande programmering och R statistiska programmeringsmiljö.

\end{changemargin}

\hypertarget{ytterligare-meriter}{%
\subsubsection{Ytterligare meriter}\label{ytterligare-meriter}}

\begin{itemize}
\tightlist
\item
  Omfattande kunskap inom statistisk programvara som SPSS, JASP och R
  statistisk programmeringsmiljö
\item
  Starka skriptkunskaper i Python (v2.7.x \& v3.x.x) och R
\item
  Betydande färdigheter i programmering och utförande av experiment med
  både E-prime och Python (dvs. PsychoPy, OpenSesame och Expyriment)
\item
  Goda kunskaper i Microsoft Word och Excel
\item
  Grundläggande programmeringskunskaper i Visual Basic, E-basic
  (E-prime), MATLAB, Bash, JavaScript och PHP
\item
  Grundläggande färdigheter i Markdown (t.ex. RMarkdown) och ~LaTeX
\end{itemize}

\hypertarget{ansvar}{%
\subsubsection{Ansvar}\label{ansvar}}

\begin{itemize}
\tightlist
\item
  Vald som ordförande för doktorandstudenternas råd vid Institutionen
  för Psykologi, Umeå universitet.
\end{itemize}

\newpage
\pagestyle{empty}

\begin{landscape}
\begin{ThreePartTable}
\begin{TableNotes}
\item \textit{Note: } 
\item  Total of clock hours: 1505 , UG = Undergraduate, G = Graduate, VT = Spring term, HT = autumn term, * = two courses.
\end{TableNotes}
\begin{longtable}[t]{l>{\raggedright\arraybackslash}p{5cm}>{\raggedright\arraybackslash}p{5cm}l>{\raggedright\arraybackslash}p{5cm}ll}
\caption{\label{tab:unnamed-chunk-5}Undervisningsansvar - en översikt över typ av undervisning, timmar etc. }\\
\toprule
Period & Subject & Type & Clock Hours & Course/Program & Level & Language\\
\midrule
\endfirsthead
\caption[]{Undervisningsansvar - en översikt över typ av undervisning, timmar etc.  \textit{(continued)}}\\
\toprule
Period & Subject & Type & Clock Hours & Course/Program & Level & Language\\
\midrule
\endhead

\endfoot
\bottomrule
\insertTableNotes
\endlastfoot
VT 2014 & Scientific communication, research methods, research ethics & Lectures, seminars, supervision & 60 & Psychological test and research methods/Statistical and Empirial Methods, the Clinical Psychologist program/the Bachelor's Program in Cognitive Science & UG & Swe\\
 &  &  &  &  &  \vphantom{14} & \\
VT, HT 2014 & Cognitive Psychology & Supervision of projects & 40 & Introduction to Psychology, the Clincial Psychologist Program & UG & Swe\\
 &  &  &  &  &  \vphantom{13} & \\
HT 2014 & Clinical Psychology & Co-supervision of thesis project & 20 & Master Thesis in Psychology, 30 ECTS, the Clinical Psychologist Program & G & Eng\\
\addlinespace
 & Applied Cognitive Science & Supervision of projects & 40 & Project in Cognitive Science, the Bachelor's Program in Cognitive Science & UG & \vphantom{1} Swe\\
 &  &  &  &  &  \vphantom{12} & \\
VT 2015 & Cognitive Psychology & Supervision of projects & 24 & Introduction to Psychology, the Clincial Psychologist Program & UG & Swe\\
 & Cognitive Psychology & Lectures, seminars, supervision of projects & 20 & Basic psychology and sport psychology, the Coaching Program & UG & Swe\\
 & Cognitive Psychology & Supervision of thesis project & 15 & Master Thesis in Cognitive Science, 15 ECTS, the Master's Program in Cognitive Science & G & Eng\\
\addlinespace
 & Applied Cognitive Science & Supervision of projects & 110 & Project in Cognitive Science, the Bachelor's Program in Cognitive Science & UG & \vphantom{1} Swe\\
 &  &  &  &  &  \vphantom{11} & \\
HT 2015 & Cognitive Psychology & Supervision of projects & 24 & Cognitive Psychology, the Clinical Psychologist Program & UG & Swe\\
 & Qualitative methods & Seminars & 20 & the Psychology of Organizations, Environment and Work, the Clinical Psychologist Program & UG & Swe\\
 &  &  &  &  &  \vphantom{10} & \\
\addlinespace
VT, HT 2016 & Cognitive Psychology & Lab demonstrations & 60 & 2:1 Cognition, the Clinical Psychologist Program & UG & Eng\\
 &  &  &  &  &  \vphantom{9} & \\
VT 2016 & Cognitive Psychology: Attention & Lectures, seminars, supervision of projects & 60 & Applied Cognitive Psychology, the Bachelor's Program in Cognitive Science & UG & Swe\\
 & Applied Cognitive Science & Supervision of projects & 110 & Project in Cognitive Science, the Bachelor's Program in Cognitive Science & UG & Swe\\
 & Qualitative and Quantitative methods & Seminars, computer labs & 60 & the Psychology of Organizations, Environment and Work, the Clinical Psychologist Program & UG & Swe\\
\addlinespace
 &  &  &  &  &  \vphantom{8} & \\
HT 2016 & Cognitive Psychology: Perception, Attention, and Consciousness & Lectures & 24 & Perception, the Bachelor's Program in Cognitive Science & UG & Swe\\
 & Clinical Psychology & Seminars & 9 & Master Thesis in Psychology, 30 ECTS, the Clinical Psychologist Program & G & Swe\\
 &  &  &  &  &  \vphantom{7} & \\
VT, HT 2017 & Cognitive Psychology & Lab demonstrations & 60 & 2:1 Cognition, the Clinical Psychologist Program & UG & Eng\\
\addlinespace
 &  &  &  &  &  \vphantom{6} & \\
VT 2017 & Cognitive Psychology, Applied: Attention & Lectures, seminar & 20 & Applied Cognitive Psychology, the Bachelor's Program in Cognitive Science & UG & Swe\\
 & Applied Cognitive Science & Supervision of projects & 40 & Project in Cognitive Science, the Bachelor's Program in Cognitive Science & UG & Swe\\
 & Qualatitive and Quantative research methods & Seminars, computer labs & 30 & the Psychology of Organizations, Environment and Work, the Clinical Psychologist Program & UG & Swe\\
 & Clinical Psychology & Seminars & 30 & Master Thesis in Psychology, 30 ECTS, the Clinical Psychologist Program & G & Swe\\
\addlinespace
 &  &  &  &  &  \vphantom{5} & \\
HT 2017 & Cognitive Psychology: Attention and Consciousness & Lectures, seminars & 24 & Perception, the Bachelor's Program in Cognitive Science & UG & Swe\\
 & Developmental Psychology & Seminar leader & 50 & Learning and teaching, pedagogy & UG & Swe\\
 &  &  &  &  &  \vphantom{4} & \\
VT 2019 & Research Methods & Workshops in SPSS & 18 & Human Resources, Special Pedagogists & G & Swe\\
\addlinespace
 & Applied Cognitive Science & Supervision of Projects & 24 & Applied Cognitive Science, the Bachelor's Program in Cognitive Science & UG & Eng\\
 & Pedagogy & Examination of Thesis & 6 & Masther Thesis in Special Pedagogy, 30 ECTS, the Specialist Pedagogy Program & G & Swe\\
 &  &  &  &  &  \vphantom{3} & \\
HT 2019 & Quantative methods, theory of science & Seminar leader, grading exams & 62 & Theory of Science and Research Methods, Pedagogy & UG & Swe\\
 & Cognitive Science & Seminar leader, grading exams & 20 & Cognitive Science - Methods, the Master's Program in Cognitive Science & G & Swe\\
\addlinespace
 & Disability Research & Lectures, workshops, examining & 40 & Theory of Science and Research Methods, Disability Research II and III & UG & Swe\\
 & Perception & Lectures & 12 & Neuropsychology and Neuroscience, the Clinical Psychologist Program & G & Swe\\
 & Developmental Psychology, and Applied Learning Disabilities* & Supervision of groups (Problem-based learning) & 62 & Cognitive Psychology, the Clinical Psychologist Program & UG & Swe\\
 &  &  &  &  &  \vphantom{2} & \\
HT, VT 2020 & Quantative methods & Lectures & 92 & Methods for Research and Development in Educational Institutions, Pedagogy & G & Swe\\
\addlinespace
 &  &  &  &  &  \vphantom{1} & \\
VT 2020 & Applied Cognitive Science & Supervision & 24 & Applied Cognitive Science, the Bachelor's Program in Cognitive Science & UG & Eng\\
 & Cognitive Science & Supervision, Examinating & 45 & Bachelor's Thesis in Cognitive Science, the Bachelor's Program in Cognitive Science & UG & Swe\\
 & Cognitive Science & Supervision & 20 & Masther's Thesis in Cognitive Science, the Master's Program in Cognitive Science & G & Swe\\
 &  &  &  &  &  & \\
\addlinespace
HT 2020 & Disability Research & Lectures, workshops, examining & 80 & Theory of Science and Research Methods, Disability Research II and III & UG & Swe\\
 & Quantative methods, theory of science & Seminar leader, grading exams & 50 & Theory of Science and Research Methods, Pedagogy & UG & Swe\\*
\end{longtable}
\end{ThreePartTable}
\end{landscape}

\end{document}