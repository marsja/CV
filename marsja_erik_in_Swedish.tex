\documentclass[]{article}
\usepackage{lscape}
\usepackage{adjustbox}
\usepackage{titling}
\usepackage{caption}
\usepackage{tabularx}
\usepackage{tikz}
\usepackage{xcolor}
\usepackage{fancyhdr}
\usepackage{lastpage}
\usepackage{titlesec}
\usepackage[scaled=.90]{helvet}% Helvetica, served as a model for arial

\pagestyle{fancy}
\fancyhf{}
\renewcommand{\headrulewidth}{0pt}
\renewcommand{\footrulewidth}{0.0pt}
\fancyfoot[CO,CE]{CV - Erik Marsja - Page \thepage \hspace{1pt} of \pageref{LastPage}}
\fancypagestyle{plain}{\pagestyle{fancy}}

\usepackage[margin=1in]{geometry}

% Font awesome
\usepackage{fontawesome}

\usepackage[T1]{fontenc}
\usepackage[utf8]{inputenc}
\usepackage[swedish]{babel}

% For tables


% Tightlist

\providecommand{\tightlist}{%
  \setlength{\itemsep}{0pt}\setlength{\parskip}{0pt}}

% URLS
\usepackage[hidelinks]{hyperref}
\usepackage{float} % here for H placement parameter

% Appendix header style

\fancypagestyle{style2}{
\fancyhf{}
\fancyhead[C]{Appendix 1}
}


% For Changing the margins (Education)

\def\changemargin#1#2{\list{}{\rightmargin#2\leftmargin#1}\item[]}
\let\endchangemargin=\endlist 

%For displaying the table in landscape format
\usepackage[absolute]{textpos}

\fancypagestyle{lscape}{% 
\fancyhf{} % clear all header and footer fields 
\fancyfoot[LE]{%
\begin{textblock}{20}(1,5){\rotatebox{90}{\leftmark}}\end{textblock}
\begin{textblock}{1}(13,10.5){\rotatebox{90}{\thepage}}\end{textblock}}
\fancyfoot[LO] {%
\begin{textblock}{1}(13,10.5){\rotatebox{90}{\thepage}}\end{textblock}
\begin{textblock}{20}(1,13.25){\rotatebox{90}{\rightmark}}\end{textblock}}
\renewcommand{\headrulewidth}{0pt} 
\renewcommand{\footrulewidth}{0pt}}

\setlength{\TPHorizModule}{1cm}
\setlength{\TPVertModule}{1cm}

% Appendix
\fancypagestyle{style2}{
\fancyhf{}
\fancyhead[C]{Appendix 1}
\renewcommand{\headrulewidth}{1pt}
}



\newcommand\secbar {
    \tikz[baseline, trim left=3.2cm] 
    {
        \fill [white] (3cm,0.1ex) rectangle +(0.2cm,1.1ex);
        \draw [gray!95, fill=gray!80] (0cm,0.1ex) rectangle (3cm,1.1ex);        
    }
}
\newcommand\subsecbar {
    \tikz[baseline, trim left=0.15cm] 
    {
        \fill [white] (2cm,0.1ex) rectangle +(0.2cm,1.1ex);
        \fill [blue!40] (0cm,0.1ex) rectangle (2cm,1.1ex);      
    }
}

\newcommand\subsubsecbar {
    \tikz[baseline, trim left=0.15cm] 
    {
        \fill [white] (1cm,0.1ex) rectangle +(0.2cm,1.1ex);
        \fill [blue!40] (0cm,0.1ex) rectangle (2cm,1.1ex);      
    }
}

\titleformat{\section}{\large}{}{0cm}{\secbar}
\titleformat{\subsection}{\large}{}{0cm}{\normalfont\sffamily\Large\bfseries\subsecbar}
\titleformat{\subsubsection}{}{}{0cm}{\normalfont\sffamily\large\bfseries}

% No first line paragraph indent
\usepackage{parskip}
\usepackage{enumitem}


\titlespacing\section{0pt}{12pt plus 4pt minus 2pt}{4pt plus 2pt minus 2pt}
\titlespacing\subsection{0pt}{12pt plus 4pt minus 2pt}{4pt plus 2pt minus 2pt}
\titlespacing\subsubsection{0pt}{12pt plus 4pt minus 2pt}{4pt plus 2pt minus 2pt}

\newcolumntype{Y}{>{\centering\arraybackslash}X}

\begin{document}

\centerline{\huge \textbf{Erik Marsja} | \textcolor{darkgray}{Curriculum Vitae}}

\vspace{2 mm}

\hrule

\begin{table}[h]
\centering
\begin{tabularx}{\textwidth}{@{}lYl@{}}
\multicolumn{3}{c}{Kandidatvägen 1, SE-907 33 Umeå, Sweden} \\\hline
 \faPhone \hspace{1 mm}  +4670-36 33 662  \hspace{1 mm}  &  & \faEnvelopeO \hspace{1 mm} \href{mailto:}{\tt \href{mailto:erik@marsja.se}{\nolinkurl{erik@marsja.se}}} \hspace{1 mm}  \\
 \faGlobe \hspace{1 mm} \href{http://marsja.se}{\tt marsja.se}   &  & \faGithub \hspace{1 mm} \href{http://github.com/marsja}{\tt marsja} \hspace{1 mm}  
 \\\hline
\end{tabularx}
\end{table}

\subsection{Nuvarande forskning}\label{nuvarande-forskning}

Jag undersöker hur plötsliga och oväntade förändringar i vår miljö
fångar vår uppmärksamhet från vad vi för närvarande är engagerade i.
Specifikt, i min avhandling undersökte jag hur plötsliga och oväntade
förändringar i auditiv, taktil och kombinationen av både auditiv och
taktil (multisensorisk) stimulering påverkar prestation i visuella
uppgifter (kategorisering av siffror och korttidsminne). Vid sidan av
vad jag studerat under min tid som doktorand så är jag även intresserad
av multisensorisk perception och hur vi skapar interna modeller baserat
på sensorisk stimulans från två, eller flera, sinnen samtidigt.

\subsection{Forskningsambition}\label{forskningsambition}

Mitt mål är att skapa min akademiska karriär inom kognitiv psykologi.
Huvudsakligen inom grundforskning men också mer tillämpad (t.ex.
multisensorala larm). Specifikt inom områdena uppmärksamhet, perception
och kognitiv kontroll. Jag strävar efter att i slutändan kunna arbeta
som en etablerad forskare inom dessa och relaterade områden (t.ex.
kognitiv neurovetenskap).

\subsection{Utbildning}\label{utbildning}

2017 \hspace{1.5cm}\textbf{Fil. Dr. i Psykologi} \hrule

\begin{changemargin}{2.3cm}{2.4cm}

Insititutionen för Psykologi, Umeå Universitet.

\textbf{Avhandlingens titel}: Attention capture by sudden and unexpected changes: a multisensory perspective. 

Tillgänglig i DiVA Databas: http://bit.ly/marsjaphd.

\textbf{Handledare}: Docent Jessica K. Ljungberg, Professor Gregory Neely, \& Fil. Dr. Patrik Hansson

\end{changemargin}

2012 \hspace{1.5cm}\textbf{Masterexamen i Kognitionsvetenskap} \hrule

\begin{changemargin}{2.3cm}{2.4cm}

Insititutionen för Psykologi, Umeå Universitet.

\textbf{Uppsatsens titel}: Attention Capture: The Impact of Change in Spatial Sound Source on Behavior. 
    
\textbf{Handledare}: Docent Jessica K. Ljungberg
    
\end{changemargin}

2011 \hspace{1.5cm}\textbf{Kandidatexamen i Kognitionsvetenskap} \hrule

\begin{changemargin}{2.3cm}{2.4cm}

Insititutionen för Psykologi, Umeå Universitet.

\textbf{Uppsatsens titel}:  Attention Capture: Studying the Distracting Effect of One’s Own Name. 

Tillgänglig i DiVA Databas: http://bit.ly/marsjabsc.
    
\textbf{Handledare}: Docent Jessica K. Ljungberg

\end{changemargin}

\subsection{Publikationer}\label{publikationer}

\subsubsection{Internationella refereegranskade
tidskrifter}\label{internationella-refereegranskade-tidskrifter}

\textbf{Marsja}, E., Neely, G., \& Ljungberg, J. K. (2018).
Investigating Deviance Distraction and the Impact of the Modality of the
To-Be-Ignored Stimuli. Experimental Psychology, 65(2), 61--70.
\url{http://doi.org/10.1027/1618-3169/a000390}

Ljungberg, K. J., Parmentier, F. B. R., Jones, D. M., \textbf{Marsja},
E., \& Neely, G. (2014). ``What's in a name?'' ``No more than when it's
mine own''. Evidence from auditory oddball distraction. Acta
Psychologica, 150C, 161--166. \url{doi:10.1016/j.actpsy.2014.05.009}.

\subsubsection{Manuskript Under Preparation/Inskickade/Under
Revidering}\label{manuskript-under-preparationinskickadeunder-revidering}

Rosa, E., \textbf{Marsja}, E., Ljungberg K.J. (Manuskript under
Preperation). Tactile warnings constitute an efficient alarm during high
mental workload in simulated flight tasks

Marsh, J.E., Vachon, F., Sörqvist, P., \textbf{Marsja}, E., Röer J.P.,
\& Ljungberg, K.J. (Under Revidering). Irrelevant vibro-tactile stimuli
produce a changing-state effect: Implications for theories of
interference in short-term memory.

\textbf{Marsja}, E., Neely, G., \& Ljungberg, K.J (Manuskript under
Preperation). Deviance distraction in the auditory and tactile
modalities after repeated exposure: differential aspects of tactile and
auditory deviants.

\textbf{Marsja}, E., Marsh, J.E., Hansson, P., \& Neely, G. (Manuskript
under Preperation). Examining the Role of Spatial Changes in Bimodal and
Uni-Modal To-Be-Ignored Stimuli and How They Affect Short-Term Memory
Processes.

\subsection{Konferenspresentationer}\label{konferenspresentationer}

\textbf{Marsja}, E., Marsh, J.E., Neely G., Hansson P., Ljungberg K.J.,
(2017, April). Domain-generality or domain-specificity of the short-term
memory: insights from a multisensory distraction paradigm. Re-thinking
the Senses Spring School, Dubrovnik, Kroatien \textbf{Poster}.

\textbf{Marsja}, E., Marsh, J.E., Neely G., Hansson P., Ljungberg K.J.,
(2016, September). Do Spatial Changes in Sounds and Vibrations Affect
Visuo-spatial and Verbal Short-Term Memory? Attention and Control:
Insights from Distraction, Workshop, University of Central Lancashire,
Preston, UK. \textbf{Inbjuden talare}.

\textbf{Marsja}, E., Marsh, J.E., Neely G., Hansson P., Ljungberg K.J.,
(2016, Juni). Spatial Change in Multisensory Distractors Impact on
Verbal and Spatial Short Term Memory. International Multisensory
Research Forum 17\textsuperscript{th} annual meeting, Suzhou, Kina.
\textbf{Muntlig presentation}.

\textbf{Marsja}, E., Neely G., Ma, L., Ljungberg K.J., (2015, August).
Cross-modality matches of intensity and attention capture dimensions of
auditory and vibrotactile stimuli. Fechner Day 2015. The
31\textsuperscript{st} Annual Meeting of the International Society for
Psychophysics, Québec, Kanada. \textbf{Poster}.

\textbf{Marsja}, E., Neely, G., Parmentier, F.B.R., Ljungberg, K.J.,
(2014, Oktober) Deviance Distraction Is Contingent on Stimuli Being
Presented within the Same Modality. Psychonomic Society's
55\textsuperscript{th} Annual Meeting. Long Beach, CA, USA.
\textbf{Poster}.

Ljungberg, K.J., Parmentier, F.B.R., \textbf{Marsja}, E., Neely, G.,
Jones, D., (2014, January). Any Tom, Dick, or Harry will do: Hearing
one's own name distracts no more than any other in a cross-modal oddball
task. Experimental Psychology Society Meeting. London, Storbrittanien.
\textbf{Poster}.

\textbf{Marsja}, E., Neely, G., Parmentier, F.B.R., Ljungberg, K.J.,
(2013, Oktober). Maintenance of the distractive effect of deviating
vibrotactile stimuli in a cross-modal oddball paradigm. the
29\textsuperscript{th} Annual meeting of the International Society of
Psychophysics, Freiburg, Tyskland. \textbf{Poster}.

\subsection{Finansiering och Bidrag}\label{finansiering-och-bidrag}

\textbf{6000 SEK} från Institituionen för Psykologi, Umeå Universitet
för deltagande i Re-thinking the Senses Spring School, Dubrovnik,
Kroatien, 2017.

\textbf{15000 SEK} from Lars Hiertas Minnesfond för projektet \emph{Är
korttidsminnet domän-generellt eller domän-specifikt? (Is short-term
memory domain-general or domain specific?)}, 2016.

\textbf{12 000 SEK} från Samhällsvetenskapliga Fakulteten, Umeå
Universitet, för deltagande i en workshop kallad Attention and Control:
Insights from Distraction and visiting a researcher at the University of
Central Lancashire, Preston, UK, 2016.

\textbf{8000 SEK} från Institituionen för Psykologi, Umeå Universitet,
för deltagande i konferensen the 17\textsuperscript{th} International
Multisensory Research Forum 15-18 Juni, Souzou, Kina, 2016.

\textbf{10 000 SEK} från JC Kempes minnesfond för projektet \emph{Is
everyday distractibility related to attention capture by vibrating
deviants?}, 2014.

\textbf{9 000 SEK} från Knut och Alice Wallenbergs Stiftelse för att
delta i konferenserna Psychonomic Society's 55\textsuperscript{th}
Annual Meeting, 20-23 November, och APCAM, 20 November, Long Beach, USA,
2014.

\textbf{6000 SEK} från Institituionen för Psykologi, Umeå Universitet
för deltagande i konferensen Fechner Day 2013 (the
29\textsuperscript{th} Annual Meeting of the International Society for
Psychophysics) 21-25 Oktober, Freiburg i.Br., Tyskland, 2013.

\subsection{Undervisningserfarenhet}\label{undervisningserfarenhet}

Främst i:

\begin{itemize}
\tightlist
\item
  Uppmärksamhet
\item
  Perception
\item
  Tillämpad kognitiv psykologi
\item
  Kognitiv psykologi
\item
  Forskningsmetodik
\end{itemize}

Föreläsningar, seminarier, laboratoriedemonstrationer, handledning av
grupprojekt (både empiriska och tillämpade projekt), och handledning av
uppsatser har givits vid Institituionen för Psykologi, Umeå Universitet.
Jag har hittills undervisat 950 klocktimmar. För en översikt över ämnen,
typ av undervisning, och antal timmar se Tabell 1 i Appendix A.

\subsubsection{Handledning av
Masterstudenter}\label{handledning-av-masterstudenter}

Ma, L. (2015). Insititutionen för Psykologi, Umeå Universitet.
``Cross-Modal Matching of Distractibility in Auditory and Tactile
Stimuli''. Masteruppsats i Kognitionsvetenskap (15 ECTS).

Blide, M. (2014). Insititutionen för Psykologi, Umeå Universitet. ``Att
orka lämna ett misshandelsförhållande: Anknytningens beydelse.
Masterupsats i Klinisk Psykologi (30 ECTS).

\subsection{Övriga färdigheter}\label{ovriga-fardigheter}

\begin{itemize}
\tightlist
\item
  Goda färdigheter i Microsoft Word och Excel
\item
  Goda färdigheter i statistisk mjukvara så som SPSS och R
\item
  Goda programmingkunskaper i Python (v2.7.x) och R
\item
  Grundläggande programmeringskunskaper i Visual Basic, E-basic
  (E-prime), och MATLAB
\item
  Grundläggande färdigheter Markdown (e.g., RMarkdown) och \LaTeX
\item
  Programmerande, och utförande, av experiment i E-prime, MATLAB, och
  Python (i.e., PsychoPy, OpenSesame, \& Expyriment)
\end{itemize}

\subsubsection{Ansvarområden}\label{ansvaromraden}

\begin{itemize}
\tightlist
\item
  Vald ordförande PsyDok, doktorandernas kårsektion, vid Institutionen
  för Psykologi, Umeå Universitet.
\end{itemize}

\begin{landscape}
\pagestyle{style2}



   \begin{table}[!htbp] \centering    \caption{Undervisning - en översikt av undervisningsyp, antal timmar, osv}    \label{}  \tiny  \begin{tabular}{@{\extracolsep{5pt}} lp{4cm}p{4cm}lp{5cm}ll} \\[-1.8ex]\hline  \hline \\[-1.8ex]  Period & Ämne & Typ & Klocktimmar & Kurs och Program & Nivå & Språk \\  \hline \\[-1.8ex]  VT 2014 & Vetenskaplig kommunikation, forskningsmetodik, forskningsetik & Föreläsningar, seminarier, handledning & 60 & Psykologiskt testning och forskningsmetodik/Statistiska och empiriska metoder, Psykologprogrammet/Kandidatprogrammet i kognitiv vetenskap & Gr & Sv \\    & Kognitiv psykologi & Handledning av projekt & 20 & Introduktion till psykologi, Psykologprogrammet & Gr & Sv \\    &  &  &   &  &  &  \\  HT 2014 & Kognitiv psykologi & Handledning av projekt & 20 & Introduktion till psykologi, Psykologprogrammet & Gr & Sv \\    & Klinisk Psykologi & Bihandledning av masteruppsats & 20 & Masteruppsats in psykologi, 30 ECTS, Psykologprogrammet & Av & Eng \\    & Tillämpad kognitionsvetenskap & Handledning av projekt & 40 & Projekt i kognitionsvetenskap, Kandidatprogrammet i kognitionsvetenskap & Gr & Sv \\    &  &  &   &  &  &  \\  VT 2015 & Kognitiv psykologi & Handledning av projekt & 24 & Introduktion till psykologi, Psykologprogrammet & Gr & Sv \\    & Kognitiv psykologi & Föreläsningar, seminarier, handledning av project & 20 & Grundläggande psykologi och idrottspsykologi, Tränarprogrammet & Gr & Sv \\    & Kognitiv psykologi & Bihandledning av masteruppsats & 15 & Masteruppsats i kognitionsvetenskap, 15 ECTS, Masterprogrammet i kognitionsvetenskap & Av & Eng \\    & Tillämpad kognitionsvetenskap & Handledning av projekt & 110 & Projekt i kognitionsvetenskap, Kandidatprogrammet i kognitionsvetenskap & Gr & Sv \\    &  &  &   &  &  &  \\  HT 2015 & Kognitiv psykologi & Handledning av projekt & 24 & Kognitiv Psykologi, Psykologprogrammet & Gr & Sv \\    & Kvalitativ forskningsmetodik & Seminarier & 20 & Organisationens psykologi, Psykologprogrammet & Gr & Sv \\    &  &  &   &  &  &  \\  VT 2016 & Kognitiv psykologi & Labdemonstrationer & 30 & 2:1 Kognition, Psykologprogrammet/Fristående kurs & Gr & Eng \\    & Kognitiv psykologi: Uppmärksamhet och Perception & Föreläsningar, seminarier, handledning av project & 60 & Tillämpad Kognitiv Psykologi, Kandidatprogrammet i kognitionsvetenskap & Gr & Sv \\    & Kvalitativ och kvantitativ forskningsmetodik & Seminarier, datorlab & 30 & Organisationens psykologi, Psykologprogrammet & Gr & Sv \\    & Tillämpad kognitionsvetenskap & Handledning av projekt & 110 & Projekt i kognitionsvetenskap, Kandidatprogrammet i kognitionsvetenskap & Gr & Sv \\    &  &  &   &  &  &  \\  HT 2016 & Kognitiv psykologi & Labdemonstrationer & 30 & 2:1 Kognition, Psykologprogrammet/Fristående kurs & Gr & Eng \\    & Kvalitativ och kvantitativ forskningsmetodik & Seminarier, datorlab & 30 & Organisationens psykologi, Psykologprogrammet & Gr & Sv \\    & Kognitiv psykologi: Uppmärksamhet och Medvetandet, Perception & Föreläsningar & 24 & Perception, Kandidatprogrammet i kognitionsvetenskap & Gr & Sv \\    & Klinisk Psykologi & Seminarier & 9 & Masteruppsats in psykologi, 30 ECTS, Psykologprogrammet & Av & Sv \\    &  &  &   &  &  &  \\  VT 2017 & Kognitiv Psykologi & Labdemonstrationer & 30 & 2:1 Kognition, Psykologprogrammet/Fristående kurs & Gr & Eng \\    & Tillämpad kognitiv psykologi: Uppmärksamhet & Föreläsningar, seminarier & 20 & Tillämpad Kognitiv Psykologi, Kandidatprogrammet i kognitionsvetenskap & Gr & Sv \\    & Tillämpad kognitionsvetenskap & Handledning av projekt & 40 & Projekt i kognitionsvetenskap, Kandidatprogrammet i kognitionsvetenskap & Gr & Sv \\    & Kvalitatiiv och kvantitativ forskningsmetodik & Seminarier, datorlab & 30 & Organisationens psykologi, Psykologprogrammet & Gr & Sv \\    & Klinisk Psykologi & Seminarier & 30 & Masteruppsats in psykologi, 30 ECTS, Psykologprogrammet & Av & Sv \\    &  &  &   &  &  &  \\  HT 2017 & Kognitiv Psykologi & Labdemonstrationer & 30 & 2:1 Kognition, Psykologprogrammet/Fristående kurs & Gr & Eng \\    & Kognitiv Psykologi: Uppmärksamhet och medvetandet  & Föreläsningar, seminarier & 24 & Perception, Kandidatprogrammet i kognitionsvetenskap & Gr & Sv \\    & Utvecklingspsykologi & Seminarieledare & 50 & Lärande och undervisning, Lärarprogrammen & Gr & Sv \\  \hline \\[-1.8ex]  \multicolumn{7}{l}{ Totalt antal klocktimmar:950, Gr = Grundnivå, Av = Avancerad nivå, Sv = Svenska, Eng = Engelska} \\  \end{tabular}  \end{table} 
\end{landscape}

\end{document}