\documentclass[]{article}
\usepackage{pdflscape}

\usepackage{everypage}

\newcommand{\Lpagenumber}{\ifdim\textwidth=\linewidth\else\bgroup
  \dimendef\margin=0 %use \margin instead of \dimen0
  \ifodd\value{page}\margin=\oddsidemargin
  \else\margin=\evensidemargin
  \fi
  \raisebox{\dimexpr -\topmargin-\headheight-\headsep-0.5\linewidth}[0pt][0pt]{%
    \rlap{\hspace{\dimexpr \margin+\textheight+\footskip}%
    \llap{\rotatebox{90}{ CV  - Erik Marsja 
- Page \thepage of \pageref{LastPage}}}}}%
\egroup\fi}
\AddEverypageHook{\Lpagenumber}%

\usepackage{adjustbox}
\usepackage{titling}
\usepackage{caption}
\usepackage{tabularx}
\usepackage{tikz}
\usepackage{xcolor}
\usepackage{fancyhdr}
\usepackage{lastpage}
\usepackage{titlesec}
\usepackage[scaled=.90]{helvet}% Helvetica, served as a model for arial

\pagestyle{fancy}
\fancyhf{}
\renewcommand{\headrulewidth}{0pt}
\renewcommand{\footrulewidth}{0.0pt}
\fancyfoot[CO,CE]{   Redog\"{o}relse f\"{o}r vetenskaplig verksamhet \\  Erik Marsja -  \thepage\hspace{1pt}(\pageref{LastPage})}
\fancypagestyle{plain}{\pagestyle{fancy}}

\usepackage[margin=1.2in]{geometry}

% Font awesome
\usepackage{fontawesome}

\usepackage[T1]{fontenc}
\usepackage[utf8]{inputenc}

% For tables


% Tightlist

\providecommand{\tightlist}{%
  \setlength{\itemsep}{0pt}\setlength{\parskip}{0pt}}

% URLS
\usepackage[hidelinks]{hyperref}
\usepackage{breakurl}
\usepackage{float} % here for H placement parameter

% Appendix header style

\fancypagestyle{style2}{
\fancyhf{}
\fancyhead[C]{Appendix 1}
}


% For Changing the margins (Education)

\def\changemargin#1#2{\list{}{\rightmargin#2\leftmargin#1}\item[]}
\let\endchangemargin=\endlist 

%For displaying the table in landscape format
\usepackage[absolute]{textpos}

\fancypagestyle{lscape}{% 
\fancyhf{} % clear all header and footer fields 
\fancyfoot[LE]{%
\begin{textblock}{20}(1,5){\rotatebox{90}{\leftmark}}\end{textblock}
\begin{textblock}{1}(13,10.5){\rotatebox{90}{\thepage}}\end{textblock}}
\fancyfoot[LO] {%
\begin{textblock}{1}(13,10.5){\rotatebox{90}{\thepage}}\end{textblock}
\begin{textblock}{20}(1,13.25){\rotatebox{90}{\rightmark}}\end{textblock}}
\renewcommand{\headrulewidth}{0pt} 
\renewcommand{\footrulewidth}{0pt}}

\setlength{\TPHorizModule}{1cm}
\setlength{\TPVertModule}{1cm}

% Appendix
\fancypagestyle{style2}{
\fancyhf{}
\fancyhead[C]{Appendix 1}
\renewcommand{\headrulewidth}{1pt}
}



\newcommand\secbar {
    \tikz[baseline, trim left=3.2cm] 
    {
        \fill [white] (3cm,0.1ex) rectangle +(0.2cm,1.1ex);
        \draw [gray!95, fill=gray!80] (0cm,0.1ex) rectangle (3cm,1.1ex);        
    }
}
\newcommand\subsecbar {
    \tikz[baseline, trim left=0.15cm] 
    {
        \fill [white] (2cm,0.1ex) rectangle +(0.2cm,1.1ex);
        \fill [blue!40] (0cm,0.1ex) rectangle (2cm,1.1ex);      
    }
}

\newcommand\subsubsecbar {
    \tikz[baseline, trim left=0.15cm] 
    {
        \fill [white] (1cm,0.1ex) rectangle +(0.2cm,1.1ex);
        \fill [blue!40] (0cm,0.1ex) rectangle (2cm,1.1ex);      
    }
}

\titleformat{\section}{\large}{}{0cm}{\secbar}
\titleformat{\subsection}{\large}{}{0cm}{\normalfont\sffamily\Large\bfseries\subsecbar}
\titleformat{\subsubsection}{}{}{0cm}{\normalfont\sffamily\large\bfseries}

% No first line paragraph indent
\usepackage{parskip}
\usepackage{enumitem}


\titlespacing\section{0pt}{12pt plus 4pt minus 2pt}{4pt plus 2pt minus 2pt}
\titlespacing\subsection{0pt}{12pt plus 4pt minus 2pt}{4pt plus 2pt minus 2pt}
\titlespacing\subsubsection{0pt}{12pt plus 4pt minus 2pt}{4pt plus 2pt minus 2pt}

\newcolumntype{Y}{>{\centering\arraybackslash}X}

\begin{document}

\centerline{\huge \textbf{Erik Marsja} | \textcolor{darkgray}{Vetenskaplig Verksamhet}}

\vspace{2 mm}

\hrule

\begin{table}[h]
\centering
\begin{tabularx}{\textwidth}{@{}lYl@{}}
\textbf{Home Address}: & &  \textbf{DOB:} 
\\Kandidatvägen 1, SE-907 33 Umeå, Sweden & &  19810526 
\\\\

 \faPhone \hspace{1 mm}  46703633662  \hspace{1 mm}  &  & \faEnvelopeO \hspace{1 mm} \href{mailto:}{\tt \href{mailto:erik@marsja.se}{\nolinkurl{erik@marsja.se}}} \hspace{1 mm}  \\
 \faGlobe \hspace{1 mm} \href{http://www.marsja.se}{\tt www.marsja.se}   &  & \faGithub \hspace{1 mm} \href{http://github.com/marsja}{\tt marsja} \hspace{1 mm}  \\
 \multicolumn{3}{c}{\emph{Languages: }Swedish, English}
 \\\hline
\end{tabularx}
\end{table}

\subsection{Forskningsprestationer och
Erfarenheter}\label{forskningsprestationer-och-erfarenheter}

\subsubsection{Multisensorisk perception, uppmärksamhet, och
korttidsminne}\label{multisensorisk-perception-uppmarksamhet-och-korttidsminne}

I min avhandling (Marsja, 2017) undersökte jag hur visuellt processande
påverkas av irrelevant auditiv och taktil stimuli. Jag undersökte
möjliga likheter och skillnader mellan distraktion av oväntade
förändrigar i auditv och taktil stimulation (Marsja, Neely, K-Ljungerg,
Under Review) och hur oväntade ljud, som presenteras bland upprepad
taktil stimulation, påverkar visuellt processande (Marsja, Neely, \&
K-Ljungberg, 2018). I de två första arbetena i min avhandling använde
jag mig av enkla visuella kategorisrings uppgifter. Dessa två studier
identifierade emellertid en kunskapslucka; hur påverkar plötsliga
förändringar i irrelevanta auditiva, taktila, eller bimodala (både
taktil och auditiv) sekvenser korttidsminnesprocesser (Marsja, Marsh,
Hansson, \& Neely, Submitted)? Hur påverkar en plötslig förändring i
stimuli spatiala plats (e.g., en förändring från höger sida av kroppen
till andra sidan) spatiala och verbala korttidsminnesprocesser?

Resultaten visade att distraktion av oväntade taktil stimuli är snarlik
distraktion av oväntade auditiv stimuli, och att en möjlig skillnad är
att effekten av oväntade taktil stimuli försvinner över tid (Marsja,
Neely, K-Ljungberg, Under Review). Vidare visade resultaten att oväntade
ljud som presenteras bland upprepade vibrationer bara fångar
uppmärksamhet om en vibration inte presentas samtidigt om ljudet. Att
utelämna en upprepad vibration räcker för att fånga uppmärksamheten
(Marsja, Neely, K-Ljungberg, 2018). Gällande korttidsminnesprocessande,
visade resultaten att en förändring i spatial plats av en irrelevant
sekvens bara stör korttidsminnesprocesser när den irrelevanta sekvensen
består av både ljud och vibrationer (bimodal betingelse). I den bimodala
betingelsen påverkades både spatiala och verbala minnesprocesser
(Marsja, Marsh, Hansson, \& Neely, Submitted).

Slutsatserna som kan dras från denna avhandling är att den centrala
mekanismen som ligger till grund för distraktion kan vara att upptäcka
oväntade förändringar i omgivningen. De oväntade förändringarna och
upprepade stimuli, i den irrelevanta sekvensen, måste presenteras inom
samma sensoriska modalitet. Resultaten från min avhandling stöder inte
idén att det kognitiva systemet bygger en neural modell som förutsätter
regelbundna sensoriska händelser från många sensoriska modaliteter
samtidigt.

Plötsliga och oväntade förändringar när det gäller rumslig lokalisering
inom irrelevanta sekvenser påverkar både verbal och spatialt
korttidsminne, när sekvensen består av både vibrationer och ljud. Detta
antyder att förändringar i bimodala sekvenser är mer utstickande och,
vidare, talar för att vissa typer av processer i korttidsminnet kan vara
domän-generella (dvs., en plötslig spatial förändring påverkade både
spatialt och verbalt korttidsminne).

Jag designade alla studier inkluderade i min avhandling, jag
programmerade även samtliga experiment för studie 1 och 3, utförde de
statistiska analyserna, tolkade resultaten, och skrev första utkasten
för samtliga studier.

\emph{Samarbetspartners: Assoc. Prof.~Jessica K-Ljungberg, Prof.~Gregory
Neely, Dr.~Patrik Hansson, and Assoc. Prof.~John E. Marsh}

Tillsammans med en internationella och nationella korttidsminnesforskare
utförde jag en serie av 3 experiment med syftet att undersöka vad som
stör visuellt korttidsminne (Marsh, Vachon, Sörqvist, Marsja, Röer, \&
K-Ljungberg, Under Revision). Specifikt, störs minne för visuella
sekvenser när en irrelevant sekvens av vibrationer presenteras
samtidigt? I denna studie använde vi oss av vibrotaktila sekvenser med
en ständig förändring (sekvensen ``hoppar'' mellan de två händerna,
vänster-höger-vänster-höger-vänster-höger;
\emph{changing-state}-sekvens), och \emph{steady-state}-sekvenser (när
alla vibrationer i sekvensen presenteras till båda händerna).

Studien visade att korttidsminne för en visuell sekvens störs mer av en
changing-state vibrotaktil sekvens jämfört med en steady-state taktil
sekvens. Effekten av en changing-state vibrotaktil sekvens är, vidare,
likartad som för changing-state sekvens bestående av ljud (Experiment
1); interferensen mellan vibrotaktila stimuli och korttidsminnet tycks
beröra återkallning av ordningen av objekt snarare än artikelidentitet
(Experiment 2); och förutsägbarheten för vibrotaktila stimuli verkar
inte modulera effektens omfattning (Experiment 3).

Slutsatsen är att resultaten talar emot idéer om flera komponenter av
korttidsminne som föreslår att störningar uppstår där
minnespresentationer överlappar varandra och delar likhet av innehåll
(e.g., Baddeley, 2012; Neath, 2000; Oberauer \& Kliegl, 2006).
Vibrerande changing-state sekvenser vars representationer sannolikt inte
delar någon likhet med innehållet i visuella verbala föremål, orsakar
störningar i serieåterkallande, förutsatt att de ändras i kontinuerligt
(dvs., changin-state). I stället stöder resultaten en enhetlig vy där
interferens uppträder inom en amodal arbetsyta där funktionellt liknande
processer kommer i konflikt (Hughes, 2014; Marsh et al., 2009).

Jag deltog i design och planering av alla experiment, programmerade alla
experiment, och skrev metod-delarna samt kommenterade och bidrog med min
kunskap om taktil information i relation till perception, uppmärksamhet,
och minne, i senare delar av manus.

\emph{Samarbetspartners: Assoc. Prof.~John E. Marsh, Prof.~Patrik
Sörqvist, Dr.~Jan P. Röer, and Assoc. Prof.~Jessica K-Ljungberg}

\subsubsection{Subjektiva och objektiva mått på
kognition}\label{subjektiva-och-objektiva-matt-pa-kognition}

Målet för detta projekt var att undersöka hur väl upplevd kognitiv
förmåga i vardagen stämmer överens med prestationsbaserade, objektiva
mått. Denna studie ämnade att 1) validera, och undersöka
faktorsturkturen, för en svensk översättning av Attentional Control
Scale (ACS; REFERENS) och 2) undersöka hur subskalorna focusing och
shifting relaterar till de exekutiva funktionerna inhibition och
shifting, respektive. Resultaten av exploratorisk och konfirmatorisk
faktoranalys visade att de två subskalorna verkar existera även i ett
svenskt stickprov, med en svensk översättning. Inga statistiskt
signifkanta korreletioner mellan inhibition eller shifting fanns
(\emph{Bayes Factors} talar för att nollhypotesen, att det inte finns
någon relation, är mer sannolik).

Jag utförde design av, och datainsamling och analys av data, samt skrev
metod-del i manus, för delstudie 1). Vidare har jag bidragit till design
av delstudie 2) och bidragit med kritiska synpunkter av sammanställande
av manus.

\emph{Samarbetspartners: Dr.~Daniel Sörman, MSc. Pia Elbe}

\newpage

\subsubsection{References}\label{references}

Baddeley, A. D. (2012). Working Memory: Theories, Models, and
Controversies. Annu. Rev.~Psychol. 63, 1--29.
\url{doi:10.1146/annurev-psych-120710-100422}.

Hughes, R. W. (2014). Auditory distraction: A duplex-mechanism account.
PsyCh J. 3, 30--41. \url{doi:10.1002/pchj.44}.

Marsh, J. E., Hughes, R. W., and Jones, D. M. (2009). Interference by
process, not content, determines semantic auditory distraction.
Cognition 110, 23--38. \url{doi:10.1016/j.cognition.2008.08.003}.

Marsh, J. E., Vachon, F., Sörqvist, P., Marsja, E., Röer, J. P., and
Ljungberg, J. K. (Under Revision). Irrelevant vibro-tactile stimuli
produce a changing-state effect: Implications for theories of
interference in short-term memory.

Marsja, E., Neely, G., and Ljungberg, J. K. (2018). Investigating
Deviance Distraction and the Impact of the Modality of the To-Be-Ignored
Stimuli. Exp. Psychol. 65. \url{doi:10.1027/1618-3169/a000390}.

Marsja, E. (2017). Attention capture by sudden and unexpected changes: a
multisensory perspective (PhD dissertation). Umeå University, Umeå.
Retrieved from
\url{http://urn.kb.se/resolve?urn=urn:nbn:se:umu:diva-141852}

Marsja, E. Elbe, P., Sörman, D. (Manuscript in Preparation). Examining
the Factor Structure of the Swedish Translation of the Attentional
Control Scale and it's relation to objective measures of attention.

Marsja, E., Marsh, J. E., Hansson, P., and Neely, G. (Submitted).
Examining the Role of Spatial Changes in Bimodal and Uni-Modal
To-Be-Ignored Stimuli and How They Affect Short-Term Memory Processes.

Marsja, E., Neely, G., and Ljungberg, J. K. (Under Review). Deviance
distraction in the auditory and tactile modalities after repeated
exposure: differential aspects of tactile and auditory deviants.

Neath, I. (2000). Modeling the effects of irrelevant speech on memory.
Psychon. Bull. Rev.~7, 403--423. \url{doi:10.3758/BF03214356}.

Oberauer, K., and Kliegl, R. (2006). Memory and Language A formal model
of capacity limits in working memory. 55, 601--626.
\url{doi:10.1016/j.jml.2006.08.009}.

\end{document}