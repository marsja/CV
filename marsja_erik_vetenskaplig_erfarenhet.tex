\documentclass[]{article}
\usepackage{pdflscape}

\usepackage{everypage}

\newcommand{\Lpagenumber}{\ifdim\textwidth=\linewidth\else\bgroup
  \dimendef\margin=0 %use \margin instead of \dimen0
  \ifodd\value{page}\margin=\oddsidemargin
  \else\margin=\evensidemargin
  \fi
  \raisebox{\dimexpr -\topmargin-\headheight-\headsep-0.5\linewidth}[0pt][0pt]{%
    \rlap{\hspace{\dimexpr \margin+\textheight+\footskip}%
    \llap{\rotatebox{90}{ CV  - Erik Marsja 
- Page \thepage of \pageref{LastPage}}}}}%
\egroup\fi}
\AddEverypageHook{\Lpagenumber}%

\usepackage{adjustbox}
\usepackage{titling}
\usepackage{caption}
\usepackage{tabularx}
\usepackage{tikz}
\usepackage{xcolor}
\usepackage{fancyhdr}
\usepackage{lastpage}
\usepackage{titlesec}
\usepackage[scaled=.90]{helvet}% Helvetica, served as a model for arial

\pagestyle{fancy}
\fancyhf{}
\renewcommand{\headrulewidth}{0pt}
\renewcommand{\footrulewidth}{0.0pt}
\fancyfoot[CO,CE]{   Redog\"{o}relse f\"{o}r vetenskaplig verksamhet \\  Erik Marsja -  DNR: AN 2.2.1-1774-18 -  \thepage\hspace{1pt}(\pageref{LastPage})}
\fancypagestyle{plain}{\pagestyle{fancy}}

\usepackage[margin=1.2in]{geometry}

% Font awesome
\usepackage{fontawesome}

\usepackage[T1]{fontenc}
\usepackage[utf8]{inputenc}
\usepackage[swedish, english]{babel}

% For tables


% Tightlist

\providecommand{\tightlist}{%
  \setlength{\itemsep}{0pt}\setlength{\parskip}{0pt}}

% URLS
\usepackage[hidelinks]{hyperref}
\usepackage{breakurl}
\usepackage{float} % here for H placement parameter

% Appendix header style

\fancypagestyle{style2}{
\fancyhf{}
\fancyhead[C]{Appendix 1}
}


% For Changing the margins (Education)

\def\changemargin#1#2{\list{}{\rightmargin#2\leftmargin#1}\item[]}
\let\endchangemargin=\endlist 

%For displaying the table in landscape format
\usepackage[absolute]{textpos}

\fancypagestyle{lscape}{% 
\fancyhf{} % clear all header and footer fields 
\fancyfoot[LE]{%
\begin{textblock}{20}(1,5){\rotatebox{90}{\leftmark}}\end{textblock}
\begin{textblock}{1}(13,10.5){\rotatebox{90}{\thepage}}\end{textblock}}
\fancyfoot[LO] {%
\begin{textblock}{1}(13,10.5){\rotatebox{90}{\thepage}}\end{textblock}
\begin{textblock}{20}(1,13.25){\rotatebox{90}{\rightmark}}\end{textblock}}
\renewcommand{\headrulewidth}{0pt} 
\renewcommand{\footrulewidth}{0pt}}

\setlength{\TPHorizModule}{1cm}
\setlength{\TPVertModule}{1cm}

% Appendix
\fancypagestyle{style2}{
\fancyhf{}
\fancyhead[C]{Appendix 1}
\renewcommand{\headrulewidth}{1pt}
}



\newcommand\secbar {
    \tikz[baseline, trim left=3.2cm] 
    {
        \fill [white] (3cm,0.1ex) rectangle +(0.2cm,1.1ex);
        \draw [gray!95, fill=gray!80] (0cm,0.1ex) rectangle (3cm,1.1ex);        
    }
}
\newcommand\subsecbar {
    \tikz[baseline, trim left=0.15cm] 
    {
        \fill [white] (2cm,0.1ex) rectangle +(0.2cm,1.1ex);
        \fill [blue!40] (0cm,0.1ex) rectangle (2cm,1.1ex);      
    }
}

\newcommand\subsubsecbar {
    \tikz[baseline, trim left=0.15cm] 
    {
        \fill [white] (1cm,0.1ex) rectangle +(0.2cm,1.1ex);
        \fill [blue!40] (0cm,0.1ex) rectangle (2cm,1.1ex);      
    }
}

\titleformat{\section}{\large}{}{0cm}{\secbar}
\titleformat{\subsection}{\large}{}{0cm}{\normalfont\sffamily\Large\bfseries\subsecbar}
\titleformat{\subsubsection}{}{}{0cm}{\normalfont\sffamily\large\bfseries}

% No first line paragraph indent
\usepackage{parskip}
\usepackage{enumitem}


\titlespacing\section{0pt}{12pt plus 4pt minus 2pt}{4pt plus 2pt minus 2pt}
\titlespacing\subsection{0pt}{12pt plus 4pt minus 2pt}{4pt plus 2pt minus 2pt}
\titlespacing\subsubsection{0pt}{12pt plus 4pt minus 2pt}{4pt plus 2pt minus 2pt}

\newcolumntype{Y}{>{\centering\arraybackslash}X}

\begin{document}

\centerline{\huge \textbf{Erik Marsja} | \textcolor{darkgray}{Vetenskaplig Verksamhet}}

\vspace{2 mm}

\hrule

\begin{table}[h]
\centering
\begin{tabularx}{\textwidth}{@{}lYl@{}}
\textbf{Home Address}: & & 
\\Kandidatvägen 1, SE-907 33 Umeå, Sweden & & 
\\\\

 \faPhone \hspace{1 mm}  +4690-786 79 59  \hspace{1 mm}  &  & \faEnvelopeO \hspace{1 mm} \href{mailto:}{\tt \href{mailto:erik.marsja@umu.se}{\nolinkurl{erik.marsja@umu.se}}} \hspace{1 mm}  \\
 \faGlobe \hspace{1 mm} \href{http://www.marsja.se}{\tt www.marsja.se}   &  & \faGithub \hspace{1 mm} \href{http://github.com/marsja}{\tt marsja} \hspace{1 mm}  \\
 \multicolumn{3}{c}{\emph{Languages: }Swedish, English}
 \\\hline
\end{tabularx}
\end{table}

\subsection{Forskningsprestationer och
Erfarenheter}\label{forskningsprestationer-och-erfarenheter}

\subsubsection{Multisensorisk perception, uppmärksamhet, och
korttidsminne}\label{multisensorisk-perception-uppmarksamhet-och-korttidsminne}

I min avhandling (Marsja, 2017) undersökte jag perceptuella och
kognitiva processer i relation till auditiv, taktil, och visuell
stimuli. Jag undersökte möjliga likheter och skillnader mellan
distraktion av plötsliga förändrigar i auditv och taktil stimulation
(Marsja, Neely, K-Ljungerg, Under Review) och hur oväntade ljud, som
presenteras bland taktil stimulation, eventuellt fångar uppmärksamheten
(Marsja, Neely, \& K-Ljungberg, 2018). I de två första arbetena i min
avhandling använde jag mig av enkla visuella kategorisrings uppgifter.
Dessa två studier identifierade emellertid en kunskapslucka; hur
påverkar plötsliga förändringar i irrelevanta auditiva, taktila, eller
bimodala (både taktil och auditiv) sekvenser minnesprocesser (Marsja,
Marsh, Hansson, \& Neely, Submitted)? Hur påverkar en plötslig
förändring i stimuli spatiala plats (e.g., en förändring från höger sida
av kroppen till andra sidan) spatiala och verbala
korttidsminnesprocesser? Resultaten visade att distraktion av oväntade
taktil stimuli är snarlik distraktion av oväntade auditiv stimuli, och
en möjlig skillnad är att effekten av oväntade taktil stimuli försvinner
över tid (Marsja, Neely, K-Ljungberg, Under Review). Vidare visade
resultaten att oväntade ljud som presenteras bland upprepade vibrationer
bara fångar uppmärksamhet om en vibration inte presentas samtidigt om
ljudet (Marsja, Neely, K-Ljungberg, 2018).

Resultaten visade slutligen att en förändring i spatial plats av en
irrelevant sekvens bara stör korttidsminnesprocesser när den irrelevanta
sekvensen består av både ljud och vibrationer (bimodal betingelse). I
den bimodala betingelsen så påverkades både spatiala och verbala
minnesprocesser (Marsja, Marsh, Hansson, \& Neely, Submitted).
Slutsatserna som kan dras från denna avhandling är att den centrala
mekanismen som ligger till grund för uppmärksamhetsfångst kan vara att
upptäcka plötsliga förändringar i omgivningen. De plötsliga
förändringarna och standard stimuli i den irrelevanta sekvensen måste
presenteras inom samma sensorisk modalitet. Det betyder att resultaten
från denna avhandling inte stöder tanken att det kognitiva systemet
bygger en neural modell som förutsätter regelbundna sensoriska händelser
från många sensoriska modaliteter samtidigt Enbart utelämnande av en
förväntad vibration kan fånga uppmärksamhet. Plötsliga och oväntade
förändringar när det gäller rumslig lokalisering inom irrelevanta
sekvenser påverkar både verbal och spatialt korttidsminne, när sekvensen
består av både vibrationer och ljud.

Jag designade alla studier inkluderade i min avhandling, jag
programmerade även samtliga experiment för studie 1 och 3, utförde de
statistiska analyserna, tolkade resultaten, och skrev första utkasten
för samtliga studier.

\emph{Samarbetspartners: Assoc. Prof.~Jessica K-Ljungberg, Prof.~Gregory
Neely, Dr.~Patrik Hansson, and Assoc. Prof.~John E. Marsh}

\paragraph{Korttidsminne}\label{korttidsminne}

Tillsammans med en rad experter i korttidsminnesforskning utförde jag en
serie av 3 experiment där vi undersökte vad som stör korttidsminnet
(Marsh, Vachon, Sörqvist, Marsja, Röer, \& K-Ljungberg, Under Revision).
Specifikt, störs korttidsminnet för en visuell sekvens när en samtidigt
presenterad irrelevant sekvens av vibrationer presenteras? Mer
specifict, sker detta endast när den vibrotaktila sekvensen medför
förändring (när sekvensen ``hoppar'' mellan de två händerna,
vänster-höger-vänster-höger-vänster-höger;
\emph{changing-state}-sekvens), och inte när sekvensen är en
\emph{steady-state}-sekvens (när alla vibrationer i sekvensen
presenteras till båda händerna).

Resultaten från de tre experimenten visade att korttidsminne för en
visuell sekvens störs mer av en changing-state vibrotaktil sekvens
jämfört med en steady-state taktil sekvens. Effekten av en
changing-state vibrotaktil sekvens är, vidare, likartad som för
changing-state auditiv sekvens (Experiment 1); Interferensen mellan
vibrotaktila stimuli och korttidsminnet tycks beröra återkallning av
ordningen av objekt snarare än artikelidentitet (Experiment 2); och
förutsägbarheten för vibrotaktila stimuli verkar inte modulera effektens
omfattning (Experiment 3).

Slutsatsen från denna serien av experiment talar emot idéer om flera
komponenter av korttidsminne som föreslår att störningar uppstår där
minnespresentationer överlappar varandra och delar likhet av innehåll
(Baddeley, 2012; Neath, 2000; Oberauer, 2009; Oberauer \& Kliegl, 2006).
Vibrerande changing-state sekvenser vars representationer sannolikt inte
delar någon likhet med innehållet i visuella verbala föremål, orsakar
störningar i serieåterkallande, förutsatt att de ändras i staten. I
stället stöder de här rapporterade resultaten en enhetlig vy där
interferens uppträder inom en amodal arbetsyta där funktionellt liknande
processer kommer i konflikt (Hughes, 2014; Jones \& Macken, 1993; Marsh
et al., 2008, 2009). Resultaten tyder på att en gemensam serieprocess
utgår från mönstret av störningar över de två innehållsförhållandena
(auditiv och vibrotaktil), ett resultat som också undergräver ett
uppmärksammat infångningskonto.

Jag deltog i design av experiment, programmerade alla experiment, och
skrev metod-delarna samt kommenterade, och bidrog med min kunskap om
taktil information i relation till perception, uppmärksamhet, och minne,
i senare delar av manus.

\emph{Samarbetspartners: Assoc. Prof.~John E. Marsh, Prof.~Patrik
Sörqvist, Dr.~Jan P. Röer, and Assoc. Prof.~Jessica K-Ljungberg}

\subsection{Subjektiva och objektiva mått på
kognition}\label{subjektiva-och-objektiva-matt-pa-kognition}

Målet för detta projekt var att undersöka hur väl upplevd kognitiv
förmåga i vardagen stämmer överens med prestationsbaserade, objektiva
mått. Denna studie ämnade att 1) validera, och undersöka
faktorsturkturen, för en svensk översättning av Attentional Control
Scale (ACS; REFERENS) och 2) undersöka hur subskalorna focusing och
shifting relaterar till de exekutiva funktionerna inhibition och
shifting, respektive. Resultaten av exploratorisk och konfirmatorisk
faktoranalys visade att de två subskalorna verkar existera även i ett
svenskt stickprov, med en svensk översättning. Inga statistiskt
signifkanta korreletioner mellan inhibition eller shifting fanns
(\emph{Bayes Factors} talar för att nollhypotesen, att det inte finns
någon relation, är mer sannolik).

\emph{Samarbetspartners: Dr.~Sörman, MSc. Pia Elbe}

\end{document}