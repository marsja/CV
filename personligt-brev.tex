\documentclass[]{article}
\usepackage{pdflscape}

\usepackage{everypage}

\newcommand{\Lpagenumber}{\ifdim\textwidth=\linewidth\else\bgroup
  \dimendef\margin=0 %use \margin instead of \dimen0
  \ifodd\value{page}\margin=\oddsidemargin
  \else\margin=\evensidemargin
  \fi
  \raisebox{\dimexpr -\topmargin-\headheight-\headsep-0.5\linewidth}[0pt][0pt]{%
    \rlap{\hspace{\dimexpr \margin+\textheight+\footskip}%
    \llap{\rotatebox{90}{ Cover Letter  - Erik Marsja 
- \thepage\hspace{1pt}(\pageref{LastPage})}}}}%
\egroup\fi}
\AddEverypageHook{\Lpagenumber}%

\usepackage{adjustbox}
\usepackage{titling}
\usepackage{caption}
\usepackage{tabularx}
\usepackage{tikz}
\usepackage{xcolor}
\usepackage{fancyhdr}
\usepackage{lastpage}
\usepackage{titlesec}
\usepackage[scaled=.90]{helvet}% Helvetica, served as a model for arial

% Added for new tabl:
\usepackage{booktabs}
\usepackage{threeparttablex}
\usepackage{longtable}


\pagestyle{fancy}
\fancyhf{}
\renewcommand{\headrulewidth}{0pt}
\renewcommand{\footrulewidth}{0.0pt}
\fancyfoot[CO,CE]{ Personligt Brev -    Erik Marsja -  DNR: dnr 2021/0400-3.1 -  \thepage\hspace{1pt}(\pageref{LastPage})}
\fancypagestyle{plain}{\pagestyle{fancy}}

\usepackage[margin=1.2in]{geometry}

% Font awesome
\usepackage{fontawesome}

\usepackage[T1]{fontenc}
\usepackage[utf8]{inputenc}

% For tables


% Tightlist

\providecommand{\tightlist}{%
  \setlength{\itemsep}{0pt}\setlength{\parskip}{0pt}}

% URLS
\usepackage[hidelinks]{hyperref}
\usepackage{breakurl}
\usepackage{float} % here for H placement parameter

% Appendix header style

\fancypagestyle{style2}{
\fancyhf{}
\fancyhead[C]{Appendix 1}
}


% For Changing the margins (Education)

\def\changemargin#1#2{\list{}{\rightmargin#2\leftmargin#1}\item[]}
\let\endchangemargin=\endlist 

%For displaying the table in landscape format
\usepackage[absolute]{textpos}

\fancypagestyle{lscape}{% 
\fancyhf{} % clear all header and footer fields 
\fancyfoot[LE]{%
\begin{textblock}{20}(1,5){\rotatebox{90}{\leftmark}}\end{textblock}
\begin{textblock}{1}(13,10.5){\rotatebox{90}{\thepage}}\end{textblock}}
\fancyfoot[LO] {%
\begin{textblock}{1}(13,10.5){\rotatebox{90}{\thepage}}\end{textblock}
\begin{textblock}{20}(1,13.25){\rotatebox{90}{\rightmark}}\end{textblock}}
\renewcommand{\headrulewidth}{0pt} 
\renewcommand{\footrulewidth}{0pt}}

\setlength{\TPHorizModule}{1cm}
\setlength{\TPVertModule}{1cm}

% Appendix
\fancypagestyle{style2}{
\fancyhf{}
\fancyhead[C]{Appendix 1}
\renewcommand{\headrulewidth}{1pt}
}



\newcommand\secbar {
    \tikz[baseline, trim left=3.2cm] 
    {
        \fill [white] (3cm,0.1ex) rectangle +(0.2cm,1.1ex);
        \draw [gray!95, fill=gray!80] (0cm,0.1ex) rectangle (3cm,1.1ex);        
    }
}
\newcommand\subsecbar {
    \tikz[baseline, trim left=0.15cm] 
    {
        \fill [white] (2cm,0.1ex) rectangle +(0.2cm,1.1ex);
        \fill [blue!40] (0cm,0.1ex) rectangle (2cm,1.1ex);      
    }
}

\newcommand\subsubsecbar {
    \tikz[baseline, trim left=0.15cm] 
    {
        \fill [white] (1cm,0.1ex) rectangle +(0.2cm,1.1ex);
        \fill [blue!40] (0cm,0.1ex) rectangle (2cm,1.1ex);      
    }
}

\titleformat{\section}{\large}{}{0cm}{\secbar}
\titleformat{\subsection}{\large}{}{0cm}{\normalfont\sffamily\Large\bfseries\subsecbar}
\titleformat{\subsubsection}{}{}{0cm}{\normalfont\sffamily\large\bfseries}

% No first line paragraph indent
\usepackage{parskip}
\usepackage{enumitem}


\titlespacing\section{0pt}{12pt plus 4pt minus 2pt}{4pt plus 2pt minus 2pt}
\titlespacing\subsection{0pt}{12pt plus 4pt minus 2pt}{4pt plus 2pt minus 2pt}
\titlespacing\subsubsection{0pt}{12pt plus 4pt minus 2pt}{4pt plus 2pt minus 2pt}

\newcolumntype{Y}{>{\centering\arraybackslash}X}

\begin{document}

\centerline{\huge \textbf{Erik Marsja} | \textcolor{darkgray}{Personligt Brev}}

\vspace{2 mm}

\hrule

\begin{table}[h]
\centering
\begin{tabularx}{\textwidth}{@{}lYl@{}}
\textbf{Address}: & & 
\\Tvistevägen 26, SE-907 36 Umeå, Swefden & & 
\\\\

 \faPhone \hspace{1 mm}  +4670-3633662  \hspace{1 mm}  &  & \faEnvelopeO \hspace{1 mm} \href{mailto:}{\tt \href{mailto:erik@marsja.se}{\nolinkurl{erik@marsja.se}}} \hspace{1 mm}  \\
 \faGlobe \hspace{1 mm} \href{http://www.marsja.se}{\tt www.marsja.se}   &  & \faGithub \hspace{1 mm} \href{http://github.com/marsja}{\tt marsja} \hspace{1 mm}  \\
 \multicolumn{3}{c}{}
 \\\hline
\end{tabularx}
\end{table}

Till den/de det berör,

Med detta brev, och bifogade handlingar, vill jag uttrycka mitt genuina
intresse för tjänsten forskare vid Väg- och Transportinstitutet (VTI),
Linköping. Efter att bedrivit grundforskning och undervisat vid
universitet finner jag det väldigt intressant att få arbeta med
forskning med ett mer tillämpat perspektiv. Detta eftersom det är en
utomordentlig möjlighet att kunna bidra mer direkt till samhället med
många av de färdigheter som jag utvecklat under min tid som forskare.
Sammantaget har min forskning fokuserat på olika aspekter av
uppmärksamhet och distraktion, hörsel, ålder och kognition. Jag har
förvärvat goda kunskaper inom multisensorisk uppmärksamhet och kognitiv
hörselvetenskap. Under min tid som postdoktor har jag erhållit kunskap
inom handikappvetenskap genom undervisning, deltagande i seminarier och
konferenser såväl som genom min egen forskning (ffa. inriktat mot
hörselnedsättning). Dessa erfarenheter skulle gagna mig i en tjänst hos
er.

Min undervisning inkluderar både kvantitativ och kvalitativ metod,
kognitiv psykologi med fokus på såväl grundforskning som tillämpning.
Jag har varit involverad i handledning av projekt där studenter arbetar
tillsammans med externa företag. Dessa projekt, inklusive uppsatser,
täcker områden som psykologi, interaktion mellan människa och dator
(t.ex. utvärdering av produktdesign), maskininlärning, insamling och
analys av data för företag samt handikappvetenskap. Majoriteten av dessa
projekt har använt sig av kvantitativ metod men även ett par med
kvalitativ metod. Dessutom har jag deltagit i flera internationella
konferenser, undervisat för internationella studenter och samarbetat med
internationella forskare och svenska företag som SOS alarm. Det gör att
jag är van att samarbeta med människor i olika typer av fora. Jag har en
bakgrund inom kognitionsvetenskap och har förutom min expertis inom
kognitiv psykologi, kognitiv hörselvetenskap, och handikappvetenskap en
generell kunskap inom alla kognitionsvetenskapliga områden, inklusive
datavetenskap (t.ex. programmering, människa-dator-interaktion och
artificiell intelligens). Detta passar er tvärvetenskapliga profil.

Jag är även intresserad av ny teknik och hur den kan användas för att
t.ex. förenkla vardagen för alla, inklusive individer med
funktionsnedsättningar. Detta intresse har, för att nämna ett exempel,
lett till ett samarbete där vi undersökte hur taktila varningssignaler
fungerar vid hög mental belastning (se bifogat CV). Min kunskap inom
multisensorisk perception och uppmärksamhet, kan göra skillnad för
individer med funktionsnedsättning exv. synnedsättning eller för
individer i situationer där t.ex. stress medför hög mental belastning.
Detta skulle vara till gagn för både samhälle och individ. På fritiden
tycker jag om att läsa, laga mat och baka surdegsbröd. Jag gillar också
att fjällvandra och umgås med min familj och vänner. Fiske är en
relativt ny hobby. Ibland tycker jag om att blogga (skriva statistik-
och programmeringsguider).

Jag ser fram emot att växa i en intellektuellt utmanande miljö och att
arbeta med fokus på interaktionen mellan människa och teknik. En tjänst
hos er skulle utöka mina färdigheter till att innefatta mer tillämpad
forskning. Jag tror starkt att min erfarenhet och meriter är till stor
nytta för VTI såväl som forskarsamhället. Tveka inte att kontakta mig
via 0703633662 eller
\href{mailto:erik@marsja.se}{\nolinkurl{erik@marsja.se}}.

Vänliga Hälsningar,

Erik Marsja

\end{document}